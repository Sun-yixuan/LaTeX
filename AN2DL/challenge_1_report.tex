\documentclass[a4paper,12pt]{article}
\usepackage{amssymb}
\usepackage{amsmath}
\usepackage{hhline}
\usepackage{hyperref}
\usepackage{mathtools}
\usepackage{bm}
\usepackage[margin=2cm]{geometry}

\usepackage{amsthm}

\usepackage{tabularx}
\usepackage{graphicx}
\usepackage{physics}
\usepackage{textcomp}


\numberwithin{equation}{section}





\begin{document}
\title{\(1^{st}\) Challenge AN2DL}
\author{Andrea Bonifacio}
\maketitle
In this report we intend to describe the steps we made during the \(1^{\text{st}}\) challenge of the AN2DL Course. It will be covered the structure of the dataset, the first steps we made, what we did right, what we did wrong and how we decided to finally tackle the problem.
\section*{Dataset}
The dataset we were given consists in \(3452\) photos of various species of plants. Inside the main folder there are \(8\) subdirectories. In every directory there are photos of the same plant. Aside from two directories, every species has about the same number of images. The first thing we noticed is the small size of the dataset, with less then \(4000\)photos there are some limits on what we can do, Every image has a size of \(96 \times 96\) pixels, with \(3\) color channels. We opted for a 90\%/10\% split for the test and validation set.
\section*{Creating the Neural Network}
Since we were dealing with images, we opted for a Convolutional Neural Network (CNN), because we thought it was the best suited model for this kind of problem. We omitted trying a model without data augmentation because at the first run it became clear that it would only lead to overfitting, a perfect accuracy with only a \(\approx 0.30\) validation accuracy.
\end{document}