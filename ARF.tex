\documentclass[a4paper,12pt]{article}
\usepackage{amssymb}
\usepackage{amsmath}
\usepackage{hhline}
\usepackage{bm}
\usepackage[margin=2cm]{geometry}
\usepackage{bigfoot}
\usepackage{amsthm}
\usepackage{tikz}
\usepackage{tabularx}
\usepackage{graphicx}
\usetikzlibrary{shapes.geometric, arrows}
\tikzstyle{startstop} = [rectangle, rounded corners, minimum width=3cm, minimum height=1cm,text centered, draw=black, fill=red!30]
\tikzstyle{io} = [trapezium, trapezium left angle=70, trapezium right angle=110, minimum width=3cm, minimum height=1cm, text centered, draw=black, fill=blue!30]
\tikzstyle{process} = [rectangle, minimum width=3cm, minimum height=1cm, text centered, draw=black, fill=orange!30]
\tikzstyle{decision} = [diamond,aspect = 2, text centered, draw=black, fill=green!30]
\tikzstyle{arrow} = [thick,->,>=stealth]
\usepackage{newunicodechar}
\newunicodechar{≠}{\ensuremath{\not =}}
\usepackage{textcomp}
\usepackage[makeroom]{cancel}

\newlength\mylength
\setlength\mylength{0.1cm}
\newcolumntype{Y}{>{\Centering\arraybackslash}X}

\AtBeginEnvironment{array}{\everymath{\displaystyle}}
\newtheoremstyle{break}
  {\partopsep}{\topsep}%  
  {\normalfont}{}
  {\bfseries}{}%
  {\newline}{}%
  \theoremstyle{break}
\newtheorem{theorem}{Theorem}[section]
\newtheorem{corollary}{Corollary}[section]
\newtheorem{proposition}{Proposition}[section]
\newtheorem{remark}[section]{Remark}
\renewcommand*{\proofname}{\textbf{Proof}}
\renewcommand\qedsymbol{$\bigstar$}
\newtheorem{definition}{Definition}[section]
\renewcommand\labelenumi{(\theenumi)}

\let\oldemptyset\emptyset
\let\emptyset\varnothing

\newcommand{\ind}{\perp\!\!\!\!\perp} 
\newcommand{\measurespace}{(X, \mathcal{M}, \mu)}
\newcommand{\sigalg}{\sigma\mbox{-algebra}}
\newcommand{\boreal}{\mathcal{B}(\mathbb{R})}
\newcommand{\code}[1]{\texttt{#1}}
\newcommand{\xdownarrow}[1]{%
  {\left\downarrow\vbox to #1{}\right.\kern-\nulldelimiterspace}
}
\newcommand{\xuparrow}[1]{%
  {\left\uparrow\vbox to #1{}\right.\kern-\nulldelimiterspace}
}
\newcommand{\arrvline}{\hfil\kern\arraycolsep\vline\kern-\arraycolsep\hfilneg}

\long\def\symbolfootnotemark[#1]#2{\begingroup%
\def\thefootnote{\fnsymbol{footnote}}\footnotetext[#1]{#2}\footnotemark[#1]\endgroup}

\long\def\symbolfootnotetext[#1]#2{\begingroup%
\def\thefootnote{\fnsymbol{footnote}}\footnotetext[#1]{#2}\endgroup}


\numberwithin{equation}{section}





\begin{document}
\title{Notes from Real and Functional Analysis}
\author{Andrea Bonifacio}
\date{\today}
\maketitle
\newpage

\section{Lesson 12/09/2022}

\subsection*{Element of set theory}
Let \(X\) be a set. Then 
\[
    \mathcal{P}(X) = \left\lbrace Y \; | \; Y \subseteq X \right\rbrace \tag{\textbf{Power Set}}
\]
Let \(I \subseteq \mathbb{R}\) be a set of indexes. A family of sets induced by \(I\) is 
\[
    \left\lbrace E_i \right\rbrace_{i \in I}, \quad E_i \subseteq X  \tag{\textbf{Family/Collection}}
\]
If \(I = \mathbb{N} \) is called a 
\[
    \left\lbrace E_n \right\rbrace_{n \in \mathbb{N}} \tag{\textbf{Sequence}}
\]
\begin{definition}
    \( \left\lbrace E_n \right\rbrace \subseteq \mathcal{P}(X) \) is monotone increasing (resp. decreasing) if 
    \[
        E_n \subseteq E_{n+1} \,\forall n \qquad (\mbox{resp. } E_n \supseteq E_{n+1} \, \forall n)
    \]
    and is written as 
    \[
        \left\lbrace E_n \right\rbrace \nearrow \qquad (\mbox{resp. }\left\lbrace E_n \right\rbrace \searrow)
    \]
\end{definition}
Given a family of sets \(\left\lbrace E_i \right\rbrace_{i \in I} \subseteq \mathcal{P}(X)\), will be often considered
\[
    \bigcup_{i \in I} E_i = \left\lbrace x \in X : \exists \; i \in I \, s.t. \, x \in E_i \right\rbrace 
\]
\[
    \bigcap_{i \in I} E_i = \left\lbrace x \in X : x \in E_i, \, \forall i \in I \right\rbrace
\]
\(\left\lbrace E_i \right\rbrace\) is said to be \textbf{disjoint} if \(E_i \cap E_j = \emptyset \; \forall i \not = j\).

Examples:
\[
    [a,b] = \bigcap_{n = 1}^{\infty} (a - \frac{1}{n}, b + \frac{1}{n}) 
\]
\[
    (a,b) = \bigcup_{n = 1}^{\infty}[a + \frac{1}{n}, b - \frac{1}{n}]
\]
\begin{definition}
    \(\left\lbrace E_n \right\rbrace \subseteq \mathcal{P}(X)\). We define 
    \[
        \limsup_{n} E_n := \bigcap_{k = 1}^{\infty} \bigl(\bigcup_{n = k}^{\infty} E_n\bigr)
    \]
    \[
        \liminf_{n} E_n := \bigcup_{k = 1}^{\infty} \bigl(\bigcap_{n = k}^{\infty} E_n\bigr)
    \]
    If these two sets are equal, then 
    \[
        \lim_n E_n = \limsup_n E_n = \liminf_n E_n
    \]
\end{definition}
\begin{proposition}
    Some limits are:
    \begin{itemize}
        \item \(\limsup_n E_n = \left\lbrace x \in X :\, x \in E_n \; \mbox{for }\infty-\mbox{many indexes }n \right\rbrace\)
        \item \(\liminf_n E_n = \left\lbrace x \in X :\, x \in E_n \; \mbox{for all but finitely many indexes }n \right\rbrace\)
        \item \(\liminf_n E_n \subseteq \limsup_n E_n\)
        \item \(\left( \liminf_n E_n\right)^C = \limsup_n E_n^C\) 
    \end{itemize}
\end{proposition}
\begin{definition}
    
    \[
    \begin{array}{ccc}
        x \in \limsup_n E_n & \Longleftrightarrow & x \in \bigcap_{k = 1}^{\infty} \left(\bigcup_{n = k}^{\infty} E_n\right) \\
        & \Longleftrightarrow & \forall k \in \mathbb{N} \, : \; \bigcup_{n = k}^{\infty} E_n \\
        & \Longleftrightarrow &  \forall k \in \mathbb{N} \; \exists n_k \geq k \, s.t. \, x \in E_{n_k}
        
    \end{array}
\]
So \(x \in \limsup_n E_n \; \Longrightarrow\) \(\begin{array}[t]{l}
    \exists m_1 = n_1 \, s.t. \, x \in E_{n_1} \\
    \exists m_2 := n_{m_1 + 1} \geq m_1 + 1 \, s.t. \, x \in E_{n_2} \\
    \vdots \\
    \exists m_k := n_{m_{k-1} + 1} \geq m_{k-1} + 1 \, s.t. \, x \in E_{n_k} \\
    \vdots \\
    x \in E_{m_1}, \ldots, E_{m_k}, \ldots 
\end{array}
\)

On the other hand, assume that \(x \in E_n\) for \(\infty\)-many indexes.
We claim that \(\forall k \in \mathbb{N} \, \exists n_k \geq k \, s.t. \, x \in E_{n_k} \, \Longleftrightarrow \, x \in \limsup_n E_n\). 
If that claim is not true, then \(\exists \bar{k} \, s.t. \, x \not \in E_n \, \forall n > \bar{k} \Longrightarrow x\) belongs at most to \(E_1, \ldots, E_{\bar{k}}\), a contradiction. 
\qed
\end{definition}
\begin{definition}
    \(\left\lbrace E_i \right\rbrace_{i \in I}\) is a \textbf{covering} of \(X\) if 
    \[
        X \subseteq \bigcup_{i \in I} E_i
    \]
A subfamily of \(E_i\) that is still a covering is called a \textbf{subcovering}
\end{definition}
\begin{definition}
    Let \(E \subseteq X\). The function \(\chi_E \, : X \rightarrow \mathbb{R}\) 
    \[
        \chi_E (x):= \begin{cases}
            1 & \mbox{if } x \in E \\
            0 & \mbox{if } x \in X\backslash E
        \end{cases}
    \]
    is called \textbf{characteristic function} of \(E\)
\end{definition}
Let \(E_1, E_2\) be sets:
\[
    \chi_{E_1 \cap E_2} = \chi_{E_1} \cdot \chi_{E_2}
\]
\[
    \chi_{E_1 \cup E_2} = \chi_{E_1} + \chi_{E_2} - \chi_{E_1 \cap E_2} 
\]
\[
    \left\lbrace E_n \right\rbrace \subseteq \mathcal{P}(X), \mbox{ disjoint}, E = \bigcup_{n = 1}^{\infty} E_n \Longrightarrow \mathcal{X_E} = \sum_{n = 1}^{\infty} \chi_{E_n}
\]
\[
    \left\lbrace E_n \right\rbrace \subseteq \mathcal{P}, P = \liminf_n E_n, Q = \limsup_n E_n \Longrightarrow \chi_P = \liminf \chi_{E_n}, \chi_Q = \limsup_n \chi_{E_n}
\]
Recall that \(\limsup_n a_n = \lim_{k \to \infty} \sup_{n \geq k} a_n\) and \(\liminf_n a_n = \lim_{k \to \infty} \inf_{n \geq k} a_n\)


Let's also check that \(\chi_Q = \limsup_n \chi_{E_n}\)
\[
    \begin{array}{ccc}   
    x \in \limsup_n E_n & \Longleftrightarrow & \chi_Q(x) = 1 \\
    & \Longleftrightarrow & \forall \, k \in \mathbb{N} \, \exists \, n_k \geq k \; s.t. \; x \in E_{n_k}
    \end{array}
    \]
If we fix \(k\) then 
\[
    \begin{array}{c}
        \sup_{n \geq k} \chi_{E_n}(x) = \chi_{E_{n_k}}(x) = 1 \\
        \lim_{k \to \infty} \sup_{n \geq k} \chi_{E_n}(x) = \limsup_n \chi_{E_n}(x) = 1
    \end{array}
\]
    Let now \(x \not \in \limsup E_n \Longleftrightarrow \chi_Q(x) = 0\).
    Then \(x\) belongs at most to finitely many \(E_n\) \(\Longrightarrow \exists \, \bar{k}\; s.t. \; x \not \in E_n, \forall n \geq \bar{k}\)
    
    If \(k \geq \bar{k}\), then \(\sup_{n \geq k} \chi_{E_n} (x) = 0 \Longrightarrow \lim_{k \to \infty} \sup_{n \geq k} \chi_{E_n}(x) = \limsup_n \chi_{E_n} (x) = 0\)

    \subsubsection*{Relations}
    Given \(X, Y\) sets, is called a \textbf{relation} of \(X\) and \(Y\) a subset of \(X \times Y\)
    \[
        R \subseteq X + Y \quad R = \left\lbrace (x,y) \, : \, x \in X, y \in Y \right\rbrace
    \]
    \[
        (x,y) \in R \Longleftrightarrow xRy
    \]
    \[
        X = \left\lbrace 0,1,2,3 \right\rbrace \quad R = \left\lbrace (0,1), (1,2), (2,1) \right\rbrace \mbox{ is a relation in } X
    \]
\begin{definition}
    A \textbf{function} from \(X\) to \(Y\) is a relation \(R\) s.t. for any element \(x\) of \(X\) \(\exists !\) element \(y\) of \(Y\) s.t. \(xRy\)
\end{definition}
\begin{definition}
    \(R\) on \(X\) is an \textbf{equivalence relation} if 
    \begin{enumerate}
        \item \(xRx\) \(\forall \; x \in X\) (\(R\) is \textbf{reflexive})
        \item \(xRy \Longrightarrow yRx\) (\(R\) is \textbf{symmetric})
        \item \(xRy, yRz \Longrightarrow xRz\) (\(R\) is \textbf{transitive})
    \end{enumerate}
    If \(R\) is an equivalence relation, the set 
    \(
        E_X := \left\lbrace y \in X \, : \, yRx \right\rbrace, \; x \in X
    \)
    is called the \textbf{equivalence class} of \(X\)
\end{definition}
\begin{definition}
    \(\frac{X}{R} := \left\lbrace E_X \, : \, x \in X \right\rbrace\) is the \textbf{quotient set}
\end{definition}
Ex: \(X = \mathbb{Z}\), let's say that \(nRm\) if \(n-m\) is even. This is an equivalence relation.
\[
    E_n = \left\lbrace \ldots, n-4, n-2, n, n+2, n+4, \ldots \right\rbrace
\]
in this case if \(n\) is even, \(E_n = \left\lbrace \mbox{even numbers} \right\rbrace\) and if \(n\) is odd, \(E_n = \left\lbrace \mbox{odd numbers} \right\rbrace\)
\subsection*{Measure theory}
\begin{definition}
    A family \(\mathcal{M} \subseteq \mathcal{P}(X)\) is called a \textbf{\(\mathbf{\sigma}\)-algebra} if 
    \begin{enumerate}
        \item \(X \in \mathcal{M}\)
        \item \(E \in \mathcal{M} \Longrightarrow E^C = X\backslash E \in \mathcal{M}\)
        \item If \(E = \bigcup_{n \in \mathbb{N}}\) and \(E_n \in \mathcal{M} \; \forall \, n\), then \(E \in \mathcal{M}\)
    \end{enumerate}
\end{definition}
If \(\mathcal{M}\) is a \(\sigalg\), \((X, \mathcal{M})\) is called \textbf{measurable space} and the sets in \(\mathcal{M}\) are called \textbf{measurable}.  
Ex: \begin{itemize}
    \item \((X, \mathcal{P}(X))\) is a measurable space
    \item Let \(X\) be a set, then \(\left\lbrace \emptyset, X \right\rbrace\) is a \(\sigalg\)
\end{itemize}
\begin{remark}
    \(\sigma\) is often used to denote the closure w.r.t. countably many operators. If we replace the countable unions with finite unions in the definition of \(\sigalg\), we obtain an \textbf{algebra}.
\end{remark}
Some \textbf{basic properties} of a measurable space \((X, \mathcal{M})\):
\begin{enumerate}
    \item \(\emptyset \in \mathcal{M}\): \(\emptyset = X^C\) and \(X \in \mathcal{M}\)
    \item \(\mathcal{M}\) is an algebra, and \(E_1, \ldots, E_n \in \mathcal{M}\)
    \[
        E_1 \cup \ldots \cup E_n = E_1 \cup \ldots \cup E_n \cup \underbrace{\emptyset}_{\in \mathcal{M}} \cup \emptyset \ldots \in \mathcal{M} 
    \]
    \item \(E_n \in \mathcal{M}\), \(\bigcap_{n \in \mathbb{N}} E_n \in \mathcal{M}\)
    \[
        \bigcap_{n \in \mathbb{N}} E_n = \biggl(\underbrace{\bigcup_{n \in \mathbb{N}} \underbrace{E_n^C}_{\in \mathcal{M}}}_{\in \mathcal{M}}\biggr)^C \qquad (\mathcal{M} \mbox{ is also closed under finite intersection})
    \]
    \end{enumerate}
\begin{itemize}
    \item \(E, F \in \mathcal{M} \Longrightarrow E \backslash F \in \mathcal{M} = E \backslash F = E \cap F^C \in \mathcal{M}\)
    \item If \(\Omega \subset X\), then the \textbf{restriction} of \(\mathcal{M}\) to \(\Omega\), written as \[\mathcal{M}\vert_{\Omega} := \left\lbrace F \subseteq \Omega: F = E \cap \Omega, \mbox{ with } E \in \mathcal{M} \right\rbrace\] is a \(\sigalg\) on \(\Omega\)
\end{itemize}
\begin{theorem}
    \(\mathcal{S} \subseteq \mathcal{P}(X)\). Then it is well defined the smallest \(\sigalg\) containing \(\mathcal{S}\), the \(\sigalg\) generated by \(\mathcal{S} := \sigma_0(\mathcal{S})\):
    \begin{itemize}
        \item \(\mathcal{S} \subseteq \sigma_0(\mathcal{S})\) and thus is a \(\sigalg\)
        \item \(\forall \sigma(\mathcal{M})\) s.t. \(\mathcal{M} \supseteq \mathcal{S}\), we have \(\mathcal{M} \supseteq \sigma_0(\mathcal{S})\)
    \end{itemize}
\end{theorem}
\begin{proof}[Proof idea]
    \[
        \mathcal{V} = \left\lbrace \mathcal{M} \subseteq \mathcal{P}(X): \mathcal{M} \mbox{ is a } \sigalg \mbox{ and }\mathcal{S} \subseteq \mathcal{M}\right\rbrace \not = \emptyset \mbox{ since } \mathcal{P}(X) \in \mathcal{V}
    \]
    We define \(\sigma_0(\mathcal{S}) = \bigcap \left\lbrace \mathcal{M} \; : \; \mathcal{M} \in \mathcal{V}\right\rbrace\), so it can be proved that this is the desired \(\sigalg\)
\end{proof}
\subsubsection*{Borel sets}
Given \((X, d)\) metric space, the \(\sigalg\) generated by the open sets is called \textbf{Borel} \(\sigalg\), written as \(\mathcal{B}(X)\). The sets in \(\mathcal{B}(X)\) are called \textbf{Borel sets}. The following sets are Borel sets:
\begin{itemize}
    \item open sets
    \item closed sets
    \item countable intersections of open sets: \(G_{\sigma}\) sets
    \item countable unions of closed sets: \(F_{\sigma}\) sets
\end{itemize}

\begin{remark}
    \(\mathcal{B}(\mathbb{R})\) can be equivalently defined as the \(\sigalg\) generated by 
    \[
        \left\lbrace (a,b): \; a,b \in \mathcal{R}, a < b \right\rbrace
    \]
    \[
        \left\lbrace (-\infty,b): \; b \in \mathcal{R} \right\rbrace
    \]
    \[
        \left\lbrace (a,+\infty ): \; a \in \mathcal{R} \right\rbrace
    \]
    \[
        \left\lbrace [a,b): \; a,b \in \mathcal{R}, a < b \right\rbrace
    \]
    \[
        \vdots
    \]
\end{remark}
\section{Lesson 14/09/2022}
\underline{Question}: What is \(\boreal\)?
Is \(\boreal \not = \mathcal{P}(\mathbb{R})\)? No.
\begin{definition}
    \((X, \mathcal{M})\) measurable space. A function \(\mu : \; \mathcal{M} \to [0, +\infty]\) is called a \textbf{positive measure} if \(\mu(\emptyset) = 0\) and if \(\mu\) is countably additive, that is 
    \[
        \forall \left\lbrace E_n \right\rbrace \subseteq \mathcal{M} \quad \mbox{disjoint}
    \]
    we have that \[
        \mu\left(\bigcup_{n=1}^{\infty}\right) = \sum_{n = 1}^{\infty} \mu(E_n) \tag*{\(\sigma\)-additivity}
    \]
\end{definition}
\begin{remark}
    a set \(A\) is countable if \(\exists \; f \; 1-1\) s.t. \(f: A\to \mathbb{N}\)
Examples: \(\mathbb{Z}, \mathbb{Q}\) are countable, while \(\mathbb{R}\) is not, also \((0,1)\) is uncountable.
\end{remark}
We always assume that \(\exists \; E \not = \emptyset, E \in \mathcal{M}\) s.t. \(\mu(E) \not = \infty\). 

If \((X,\mathcal{M})\) is a measurable space, and \(\mu\) is a measure on it, then \((X, \mathcal{M}, \mu)\) is a measure space.

Then:
\begin{enumerate}
    \item \(\mu\) is \textbf{finitely additive}: 
    \[
        \forall \; E,F \in \mathcal{M}, \mbox{ with } E \cap F \not = \emptyset \Longrightarrow
    \mu(E \cup F) = \mu(E) + \mu(F)
    \]
    \item the \textbf{excision property}
    \[
        \forall \; E, f \in \mathcal{M}, \mbox{ with } E \subset F \mbox{ and } \mu(E) < +\infty \Longrightarrow \mu(F\backslash E) = \mu(F) - \mu(E)
    \]
    \item \textbf{monotonicity}
    \[
        \forall \; E, F \in \mathcal{M}, \mbox{ with } E \subset F \Longrightarrow \mu(E) \leq \mu(F)
    \]
    \item if \(\Omega \in \mathcal{M}\) then \((\Omega, \mathcal{M}\vert_{\Omega}, \mu\vert_{\mathcal{M}\vert_{\Omega}})\) is a measure space
\end{enumerate}
\begin{proof}
     \begin{enumerate}
        \item \(E_1 = E, E_2 = F, E_3 = \ldots = E_n = \ldots = \emptyset\) 
        This is a disjoint sequence \(\Longrightarrow\) by \(\sigma\)-additivity. 
        \[
            \mu(E \cup F) = \mu\left(\bigcup_{n} E_n\right) = \sum_n \mu(E_n) = \mu(E) + \mu(F) + \underbrace{\mu(E_k)}_{= \mu(\emptyset)}
        \] 
        \item \(E \subset F\), so \(F = E \cup (F \backslash E)\) and this is disjoint \(\overset{(i)}{\Longrightarrow} \mu(F) = \mu(E) + \mu(F\backslash E)\), and since \(\mu(E) < \infty\), the property follows.
        \item \(E \subset F \Longrightarrow \mu(F) = \mu(E) + \underbrace{\mu(F\backslash E)}_{\geq 0} \geq \mu(E)\)
        \item 
     \end{enumerate}
    \end{proof}
\begin{definition}
    \((X, \mathcal{M}, \mu)\) measure space. 
    \begin{itemize}
    \item If \(\mu(X) < +\infty\), we say that \(\mu\) is \textbf{finite}.

    \item If \(\mu (X) = +\infty\), and \(\exists \; \left\lbrace E_n \right\rbrace \subset \mathcal{M}\) s.t. \(X = \bigcup_n E_n\) and each \(E_n\) has finite measure, then we say that \(\mu\) is \(\sigma\)-finite. 

    \item If \(\mu(X) = 1\) we say that \(\mu\) is a \textbf{probability measure}.
    \end{itemize}
\end{definition}
    Some examples:
    \begin{itemize}
        \item Trivial Measure: \((X, \mathcal{M})\) measurable space. \(\mu : \mathcal{M} \to [0, \infty]\) defined by \(\mu(E) = 0 \quad \forall \; E \in \mathcal{M}\) 
        \item Counting Measure: \((X, \mathcal{P}(X))\) measurable space. We define 
        \[
            \mu_C :  \mathcal{P}(X) \to [0, \infty], \quad \mu_C (E) = \begin{cases}
                n & \mbox{if } E \mbox{ has } n \mbox{ elements} \\
                \infty &  \mbox{if } E \mbox{ has } \infty \mbox{-many elements} 
            \end{cases}
        \]
        \item Dirac Measure: \((X, \mathcal{P}(X))\) measurable space, \(t \in X\). We define 
        \[
            \delta_t  :  \mathcal{P}(X) \to [0, +\infty], \quad \delta_t(E) = \begin{cases}
                1 & \mbox{if } t \in E\\
                0 & \mbox{otherwise}
            \end{cases}
        \]
    \end{itemize}
\underline{Continuity of the measure along monotone sequences}

\((X, \mathcal{M}, \mu)\) measure space
\begin{enumerate}
    \item \(\left\lbrace E_i \right\rbrace \subset \mathcal{M}, \; E_i \subseteq E_{i+1} \; \forall i\) and let \[
        E = \bigcup_{i = 1}^{\infty} E_i = \lim_i E_i
    \]
    Then:
    \[
        \mu(E) = \lim_i \mu(E_i)
    \]
    \item \(\left\lbrace E_i \right\rbrace \subset \mathcal{M}, \; E_{i+1} \subseteq E_{i} \; \forall i\) and let \(E = \bigcap_{i = 1}^{\infty} E_i = \lim_i E_i\).
\end{enumerate}
\begin{proof}
    \begin{enumerate}
        \item if \(\exists \; i\) s.t. \(\mu(E_i) = +\infty\), then is trivial. Assume then that every \(E_i\) has a finite measure, so that \(E = \bigcup_{i=1}^{\infty} E_i = \bigcup_{i=0}^{\infty}(E_{i+1}\backslash E_i)\) with \(E_0 = \emptyset\).
        
        So, by \(\sigma\)-additivity \[\mu(E) = \mu\left(\bigcup_{i=0}^{\infty}(E_{i+1}\backslash E_i)\right) = \]
        \[
            = \sum_{i = 0}^{\infty} \mu(E_{i + 1} \backslash E_i) \overset{(excision)}{=} \sum_{i=0}^{\infty}\left(\mu(E_{i+1} - \mu(E_i))\right) = 
        \]
        \[
            \overset{(telescopic \; series)}{=} \lim_n \mu(E_n) - \underbrace{\mu(E_0)}_{= 0} = \lim_n \mu(E_n)
        \]
        \item For simplicity, suppose \(\tau = 1\), and define \(F_k = E_i\backslash E_k\) 
        \[
            \left\lbrace E_k \right\rbrace \searrow \Longrightarrow \left\lbrace F_k \right\rbrace \nearrow
        \]
        \[
            \mu(E_i) = \mu(E_k) + \mu(F_k) \mbox{ and } \bigcup_k F_k = E_i \backslash (\bigcap_k E_k)
        \]
        \[
            \mu(E_i) = \mu(\bigcup_k F_k) + \underbrace{\mu(\bigcap_k E_k)}_{\mu(E)} =
        \]
        \[
            \overset{(i)}{=} \lim_k \mu(F_k) + \mu(E) = \lim_k \left(\mu(E_i) - \mu(E_k)\right) + \mu(E)
        \]
        Since \(\mu(E_i) < \infty\) we can subtract it from both sides
        \[
            0 = -\lim_k \mu(E_k) + \mu(E)
        \]
    \end{enumerate}
\end{proof}
Counterexample: given \((\mathcal{N}, \mathcal{P}(\mathbb{N}), \mu_C)\) measure space. Let \(E_n = \left\lbrace n, n+1, n+2, \ldots\right\rbrace\). In this case \(\mu_C (E_n) = +\infty, E_{n+1} \subseteq E_n \forall \; n\), but \(\bigcap_n E_n = \emptyset \Longrightarrow \mu\left(\bigcap_n E_n\right) = 0 \)
\begin{theorem}[\(\sigma\)-subadditivity of the measure]
\((X, \mathcal{M}, \mu)\) is a measure space. \(\forall \left\lbrace E_n \right\rbrace \subseteq \mathcal{M}\) (not necessarily disjoint): \(\mu\left(\bigcup_n E_n\right) \leq \sum_n \mu(E_n)\)
\end{theorem}
\begin{proof}
    \(E_1, E_2 \in \mathcal{M}\) and also \(E_1 \cup E_2 = E_1 \cup (E_2 \backslash E_1)\) disjoint sets.
    \[
        \mu(E_1 \cup E_2) = \mu(\underbrace{E_2 \backslash E_1}_{\subseteq E_2}) \overset{(monotonicity)}{\leq} \mu(E_1) + \mu(E_2)
    \]
    that means that we have the subadditivity for finitely many sets.
    \[
        A = \bigcup_{n=1}^{\infty} E_n, \quad A_k = \bigcup_{n = 1}^{k} E_n
    \]
    \[
        \left\lbrace A_k \right\rbrace \nearrow, \; A_{k+1} \supseteq A_k, \; \lim_k A_k = A
    \]
    \[
        \mu\left(\bigcup_{n = 1}^{\infty} E_n\right) \overset{(continuity)}{=} \lim_k \mu(A_k) = \lim_k \mu \left(\bigcup_{n=1}^{k} E_n\right) \leq
        \]
        \[
            \leq \lim_k \sum_{n=1}^k \mu(E_n) = \sum_{n = 1}^{\infty} \mu(E_n)
        \]
\end{proof}
Exercise: \((X, \mathcal{M})\) measurable space. \(\mu : \mathcal{M} \to [0, +\infty]\) s.t. \(\mu\) is finitely additive, \(\sigma\)-subadditive and \(\mu(\emptyset) = 0\) \(\Longrightarrow\) \(\mu\) is \(\sigma\)-additive, and hence is a measure.  

Exercise: the Borel-Cantelli lemma states that, given \((X, \mathcal{M}, \mu)\) and \(\left\lbrace E_n \right\rbrace \subseteq \mathcal{M}\). Then
\[
    \sum_{n=0}^{\infty} \mu(E_n) < \infty \Longrightarrow \mu(\limsup_n E_n) = 0
\]
It can be phrased as: \begin{quote}
    If the series of the probability of the events \(E_n\) is convergent, then the probability that \(\infty\)-many events occur is \(0\)
\end{quote}
\begin{proof}
    The thesis is: \[\mu(\limsup_n E_n) = \mu\biggl(\bigcap_{n=1}^{\infty} \underbrace{\bigcup_{k \geq n} E_k}_{A_n := \bigcup_{k\geq n}E_k}\biggr)\]
    Is it true that \(\left\lbrace A_n \right\rbrace \searrow\)? Yes.
    \[
         A_{n+1} = \bigcup_{k \geq n+1}  E_k \subseteq \bigcup_{k \geq n} E_k = A_n
    \]
    Does some \(A_n\) have a finite measure? 
    \[
        \mu(A_n) = \mu\left(\bigcup_{k \geq n} E_k\right) \leq \sum_{k \geq n} \mu(E_k) < \infty
    \]
    by assumption. Therefore, we can use the continuity along decreasing sequences: 
    \[
        \mu(\limsup_n E_n) = \lim_n \mu(A_n) = \lim_n \mu \left(\bigcup_{k \geq n} E_k\right) \overset{\sigma-sub.}{\leq} \lim_n \sum_{k=n}^{\infty} \mu(E_k) = 0
    \]
\end{proof}
\subsubsection*{Sets of \(0\) measure}
\((X, \mathcal{M}, \mu)\) measure space.
\begin{itemize}
    \item \(N \subseteq X\) is a set of \(0\) measure if \(N \in \mathcal{M}\) and \(\mu(N) = 0\)
    \item \(E \subseteq X\) is called \textbf{negligible set} if \(\exists \; N \in \mathcal{M}\) with \(0\) measure s.t. \(E \subseteq N\) (\(E\) does not necessarily stays in \(\mathcal{M}\))
\end{itemize} 
\begin{definition}
    \((X, \mathcal{M}, \mu)\) measure space s.t. every negligible set is measurable (and hence of \(0\) measure), then \(\measurespace\) is said to be a \textbf{complete measure space}.

    A measure space may not be complete. However, let 
    \[
        \overline{\mathcal{M}} := \left\lbrace E \subseteq X : \exists\; F, G \in \mathcal{M} \mbox{ with } F\subseteq E \subseteq G \mbox{ and } \mu(G\backslash F) = 0\right\rbrace
    \]
    Clearly \(\mathcal{M} \subseteq \overline{\mathcal{M}}\). For \(E \in \overline{\mathcal{M}}\), take \(F\) and \(G\) as above and let \(\bar{\mu}(E) = \bar{\mu}(F)\) then \(\bar{\mu}\vert_{\mathcal{M}} = \mu\), and moreover:
\end{definition}
\begin{theorem}
    \(\measurespace\) is a complete measure space. Let's observe that \(\bar{\mu}\) is well defined: let \(E \subseteq X\) and \(F_1,F_2, G_1, G_2\) s.t. \(F_i \subset E \subset G_i \quad i = 1,2\). Then \(\mu(G_i\backslash F_i) = 0\). Now we have to check that \(\mu(F_1) = \mu(F_2)\). 

    Since \[
        F_1 \backslash F_2 \subseteq E\backslash F_2 \subseteq G_2 \backslash F_2
    \] 
    and \(G_2 \backslash F_2\) has \(0\) measure \(\Longrightarrow \mu(F_1 \backslash F_2) = 0\). Then \(F_1 = (F_1 \backslash F_2) \cup (F_1 \cap F_2) \Longrightarrow \mu(F_1) = \mu(F_1 \cap F_2).\) In the same way, \(\mu(F_2) = \mu(F_1 \cap F_2)\)
\end{theorem}
\section{Lesson 15/09/2022}
The elements of \(\overline{\mathcal{M}}\) are sets of the type \(E \cup N\), with \(E \in \mathcal{M}\) and \(\bar{\mu}(N) = 0\).
\subsubsection*{Outer measure}
We wish to defin
\section{Lesson 22/09/2022}
\begin{enumerate}
    \item  \(((X, \mathcal{M}))\) is a measurable space obtained by means of an outer measure. Ex: \((\mathbb{R}^n, \mathcal{L}(\mathbb{R}^n))\), \((Y, d_y)\) metric space
    If \(X \to Y\) is (Lebesgue) measurable \(\Longleftrightarrow\) \((\mathcal{M}, \mathcal{B}(Y))\) is measurable
    \item \((X, d_X), (Y, d_Y)\) are metric spaces \(\longrightarrow (X, \mathcal{B}(X))\)
    If \(X \to Y\) are borel measurable \(\Longleftrightarrow (\mathcal{B}(X), \mathcal{B}(Y)) \)measurable
\end{enumerate}
\begin{remark}
     \(f\) is Lebesgue measurable if the continuity of the borel set is a Lebesgue-measurable set.
\end{remark}
\begin{proposition}
    \begin{enumerate}
        \item \((X, d_X), (Y, d_Y)\) metric spaces. If \(X \to Y\) is continuous, then is Borel measurable
        \item \((Y, d_Y)\) metric space. If \(\mathbb{R}^n \to Y\) is continuous, then it is a Lebesgue measure.
    \end{enumerate}
\end{proposition}
\begin{proof}
    \begin{enumerate}
        \item \(f\) is continuous \(\Longleftrightarrow f^{-1}(A)\) is open \(\forall \; A \in Y\)
        open \(\Longrightarrow\) \(f^{-1}(A) \in \mathcal{B}(Y) \; \forall \; A \in Y\) open
        Since \(\mathcal{B}(Y) = \sigma_0(\mbox{open sets})\) by proposition \(1\) thus implies that \(f\) is Borel measurable
        \item \(f\) is continuous \(\Longrightarrow\) \(f\) is Borel measurable
        mancano pezzi namely \(f\) is Lebesgue measurable
    \end{enumerate}
\end{proof}
\begin{proposition}
    \((X, \mathcal{M})\) measurable space, \((X, d_Y), (Y, d_Y)\) metric spaces. 
    if \(f: X \to Y\) is \(\mathcal{M}, \mathcal{B}(Y)\)-measurable and \(g : Y \to Z\) is continuous \(\Longrightarrow\) \(g \circ f : x \to Z\) is \(\mathcal{M}, \mathcal{B}(Y)\)-measurable
\end{proposition}
\begin{proposition}
    \((X, \mathcal{M})\) measurable space 
    Let \(\Phi : \mathbb{R}^n \to Y\) be continuous where \((Y, d_Y)\) is a metric space. Then \(h: X\to Y\) defined by \(h(x) = \Phi(u(x), boh)\) is \(\mathcal{M}, \mathcal{B}(Y)\)-measurable.
\end{proposition}
\begin{proof}
    Define \(f: X \to \mathbb{R}^n\), \(f(x) = u(x), v(x)\). By def \(h = \Phi \circ f\) by prop 3 if we show that \(f\) is measurable, then \(h\) is measurable. It can be proved that \[\mathcal{B}(\mathbb{R}^2) = \sigma_0 \left(\left\lbrace (a_1, b_1) \times (a_2, b_2): a,b \in \mathbb{R}\right\rbrace\right)\]
    pezzi
    \(f^{-1}(\mathcal{R} \in \mathcal{M}) \quad \forall \mbox{open rectangle in }\mathcal{R}^2\)
    \(R = I \times J\)
    \(F^{-1} = \left\lbrace x \in X \right\rbrace\) 
\end{proof}
\begin{remark}
    roba 
    \[
        g(x) = \begin{cases}
            x & \mbox{where } x \geq 0 \\
            0 & \mbox{where } x < 0
        \end{cases}
    \]
\end{remark}
cosine 
\((X,\mathcal{M})\) measurable space, then such a  function f is measurable iff
\[
    f^{-1}(a, +\infty)] \in \mathcal{M} \quad \forall a \in \mathcal{R}
\]
LEt now \(\left\lbrace f_n \right\rbrace\) be a Sequence of measurable functions from \(X\) to \(\bar{\mathcal{R}}\). Then we define \[
    (\inf_n f_n)(x) = \inf_n f_n(x)
\]
\[
    (\sup_n f_n)(x) = \sup_n f_n(x)
\]
\[
    (\liminf_n f_n)(x) = \liminf_n f_n(x)
\]
\[
    (\limsup_n f_n)(x) = \limsup_n f_n(x)
\]
\[
    (\lim_n f_n)(x) = \lim_n f_n(x) \quad \mbox{if the limit exists}
\]
\begin{proposition}
    \((X, \mathcal{M})\) measurable space, \(f_n : X \to \bar{\mathcal{R}}\) measurable, then 
    \(\sup \inf \liminf \limsup\) of \(f_n\) are measurable, in particular if \(\lim f_n\) exists, then \(f\) is measurable
\end{proposition}
\begin{proof}
    \((\sup f_n)^{-1} ((a, \infty]) = \left\lbrace x \in X : \sup f_n(x) > a \right\rbrace\) (manca pezzi)
    \[
        \bigcup \left\lbrace x \in X : f_n(x) > a \right\rbrace
    \]
    Then \((\sup f_n)^{-1} ((a, \infty])\) is measurable, cose da aggiungere
Noe the limsup
\[\limsup_n f_n = \lim_n (\sup_{k > n} f_n(x))\]
cose cose 
\end{proof}
\subsection*{Simple functions}
\begin{definition}
    \((X, \mathcal{M})\) measurable space. A measurable function s: \(X \to \bar{\mathcal{R}}\) is said to be simple if \(s(X)\) is a finite set 
    altre cose 
    Then \(s(x) = \sum_{n = 1} a_n \chi_{E_n}(x)\) where \(E_n\) is a measurable set 
    sistemare. 
\end{definition}
\underline{Particular case}: if s:\(\mathbb{R} \to \bar{\mathbb{R}}\) and each \(E_n\) is a finite union of intervals, then \(s\) is said to be a STEP FUNCTION.

The idea is to approximate functions with simple functions.
\begin{theorem}
    \((X,\mathcal{M})\) measurable space, \(f: X \to [0, \infty]\) measurable. Then \(\exists\) a sequence \(\left\lbrace s_n \right\rbrace\) of simple functions s.t. 
    \[
        0 \leq s_1 \leq \ldots \leq s_n \leq \ldots \leq f \quad \mbox{pointwise}
    \]
    and \(s_n(x) \to f(x)\)
Moreover if f is bounded then \(s_n \to f\) uniformly on \(X\) as \(n \to \infty\)
\end{theorem}
\begin{proof}[f is bounded]
    Fix \(n \in \mathbb{N}\) and divide \([0,n)\) in \(n \cdot 2^n\) intervals called \(I_j = [a_j,b_j)\) with lenght \(\frac{1}{2^n}\)

    Let \(E_0 = f^{-1}([n, \infty)), E_j = f^{-1}([a_j, b_j))\) for \(j = 1, \ldots, n\cdot 2^n\)
    
    We let Array

    Namely we define 
    \[
    s_n (x) = n\chi_{E_0}(X) + \sum_{j =1}^{n \cdot 2^n} a_j \chi_{E_j}(x)    
    \]
    Then \(s_n \leq s_{n+1}\) by contradiction

    Then any \(x \in X\) stays in \(f^{-1}([a_j, b_j))\) for some j \(\Longrightarrow \) 
\end{proof}
\end{document}

 