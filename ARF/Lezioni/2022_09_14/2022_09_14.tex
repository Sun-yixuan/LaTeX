\section{Lesson 14/09/2022}
\underline{Question}: What is \(\boreal\)?
Is \(\boreal \not = \mathcal{P}(\mathbb{R})\)? No.
\begin{definition}
    \((X, \mathcal{M})\) measurable space. A function \(\mu : \; \mathcal{M} \to [0, +\infty]\) is called a \textbf{positive measure} if \(\mu(\emptyset) = 0\) and if \(\mu\) is countably additive, that is 
    \[
        \forall \left\lbrace E_n \right\rbrace \subseteq \mathcal{M} \quad \mbox{disjoint}
    \]
    we have that \[
        \mu\left(\bigcup_{n=1}^{\infty}\right) = \sum_{n = 1}^{\infty} \mu(E_n) \tag*{\(\sigma\)-additivity}
    \]
\end{definition}
\begin{remark}
    a set \(A\) is countable if \(\exists \; f \; 1-1\) s.t. \(f: A\to \mathbb{N}\)
Examples: \(\mathbb{Z}, \mathbb{Q}\) are countable, while \(\mathbb{R}\) is not, also \((0,1)\) is uncountable.
\end{remark}
We always assume that \(\exists \; E \not = \emptyset, E \in \mathcal{M}\) s.t. \(\mu(E) \not = \infty\). 

If \((X,\mathcal{M})\) is a measurable space, and \(\mu\) is a measure on it, then \((X, \mathcal{M}, \mu)\) is a measure space.

Then:
\begin{enumerate}
    \item \(\mu\) is \textbf{finitely additive}: 
    \[
        \forall \; E,F \in \mathcal{M}, \mbox{ with } E \cap F \not = \emptyset \Longrightarrow
    \mu(E \cup F) = \mu(E) + \mu(F)
    \]
    \item the \textbf{excision property}
    \[
        \forall \; E, f \in \mathcal{M}, \mbox{ with } E \subset F \mbox{ and } \mu(E) < +\infty \Longrightarrow \mu(F\backslash E) = \mu(F) - \mu(E)
    \]
    \item \textbf{monotonicity}
    \[
        \forall \; E, F \in \mathcal{M}, \mbox{ with } E \subset F \Longrightarrow \mu(E) \leq \mu(F)
    \]
    \item if \(\Omega \in \mathcal{M}\) then \((\Omega, \mathcal{M}\vert_{\Omega}, \mu\vert_{\mathcal{M}\vert_{\Omega}})\) is a measure space
\end{enumerate}
\begin{proof}
     \begin{enumerate}
        \item \(E_1 = E, E_2 = F, E_3 = \ldots = E_n = \ldots = \emptyset\) 
        This is a disjoint sequence \(\Longrightarrow\) by \(\sigma\)-additivity. 
        \[
            \mu(E \cup F) = \mu\left(\bigcup_{n} E_n\right) = \sum_n \mu(E_n) = \mu(E) + \mu(F) + \underbrace{\mu(E_k)}_{= \mu(\emptyset)}
        \] 
        \item \(E \subset F\), so \(F = E \cup (F \backslash E)\) and this is disjoint \(\overset{(i)}{\Longrightarrow} \mu(F) = \mu(E) + \mu(F\backslash E)\), and since \(\mu(E) < \infty\), the property follows.
        \item \(E \subset F \Longrightarrow \mu(F) = \mu(E) + \underbrace{\mu(F\backslash E)}_{\geq 0} \geq \mu(E)\)
        \item 
     \end{enumerate}
    \end{proof}
\begin{definition}
    \((X, \mathcal{M}, \mu)\) measure space. 
    \begin{itemize}
    \item If \(\mu(X) < +\infty\), we say that \(\mu\) is \textbf{finite}.

    \item If \(\mu (X) = +\infty\), and \(\exists \; \left\lbrace E_n \right\rbrace \subset \mathcal{M}\) s.t. \(X = \bigcup_n E_n\) and each \(E_n\) has finite measure, then we say that \(\mu\) is \(\sigma\)-finite. 

    \item If \(\mu(X) = 1\) we say that \(\mu\) is a \textbf{probability measure}.
    \end{itemize}
\end{definition}
    Some examples:
    \begin{itemize}
        \item Trivial Measure: \((X, \mathcal{M})\) measurable space. \(\mu : \mathcal{M} \to [0, \infty]\) defined by \(\mu(E) = 0 \quad \forall \; E \in \mathcal{M}\) 
        \item Counting Measure: \((X, \mathcal{P}(X))\) measurable space. We define 
        \[
            \mu_C :  \mathcal{P}(X) \to [0, \infty], \quad \mu_C (E) = \begin{cases}
                n & \mbox{if } E \mbox{ has } n \mbox{ elements} \\
                \infty &  \mbox{if } E \mbox{ has } \infty \mbox{-many elements} 
            \end{cases}
        \]
        \item Dirac Measure: \((X, \mathcal{P}(X))\) measurable space, \(t \in X\). We define 
        \[
            \delta_t  :  \mathcal{P}(X) \to [0, +\infty], \quad \delta_t(E) = \begin{cases}
                1 & \mbox{if } t \in E\\
                0 & \mbox{otherwise}
            \end{cases}
        \]
    \end{itemize}
\underline{Continuity of the measure along monotone sequences}

\((X, \mathcal{M}, \mu)\) measure space
\begin{enumerate}
    \item \(\left\lbrace E_i \right\rbrace \subset \mathcal{M}, \; E_i \subseteq E_{i+1} \; \forall i\) and let \[
        E = \bigcup_{i = 1}^{\infty} E_i = \lim_i E_i
    \]
    Then:
    \[
        \mu(E) = \lim_i \mu(E_i)
    \]
    \item \(\left\lbrace E_i \right\rbrace \subset \mathcal{M}, \; E_{i+1} \subseteq E_{i} \; \forall i\) and let \(E = \bigcap_{i = 1}^{\infty} E_i = \lim_i E_i\).
\end{enumerate}
\begin{proof}
    \begin{enumerate}
        \item if \(\exists \; i\) s.t. \(\mu(E_i) = +\infty\), then is trivial. Assume then that every \(E_i\) has a finite measure, so that \(E = \bigcup_{i=1}^{\infty} E_i = \bigcup_{i=0}^{\infty}(E_{i+1}\backslash E_i)\) with \(E_0 = \emptyset\).
        
        So, by \(\sigma\)-additivity \[\mu(E) = \mu\left(\bigcup_{i=0}^{\infty}(E_{i+1}\backslash E_i)\right) = \]
        \[
            = \sum_{i = 0}^{\infty} \mu(E_{i + 1} \backslash E_i) \overset{(excision)}{=} \sum_{i=0}^{\infty}\left(\mu(E_{i+1} - \mu(E_i))\right) = 
        \]
        \[
            \overset{(telescopic \; series)}{=} \lim_n \mu(E_n) - \underbrace{\mu(E_0)}_{= 0} = \lim_n \mu(E_n)
        \]
        \item For simplicity, suppose \(\tau = 1\), and define \(F_k = E_i\backslash E_k\) 
        \[
            \left\lbrace E_k \right\rbrace \searrow \Longrightarrow \left\lbrace F_k \right\rbrace \nearrow
        \]
        \[
            \mu(E_i) = \mu(E_k) + \mu(F_k) \mbox{ and } \bigcup_k F_k = E_i \backslash (\bigcap_k E_k)
        \]
        \[
            \mu(E_i) = \mu(\bigcup_k F_k) + \underbrace{\mu(\bigcap_k E_k)}_{\mu(E)} =
        \]
        \[
            \overset{(i)}{=} \lim_k \mu(F_k) + \mu(E) = \lim_k \left(\mu(E_i) - \mu(E_k)\right) + \mu(E)
        \]
        Since \(\mu(E_i) < \infty\) we can subtract it from both sides
        \[
            0 = -\lim_k \mu(E_k) + \mu(E)
        \]
    \end{enumerate}
\end{proof}
Counterexample: given \((\mathcal{N}, \mathcal{P}(\mathbb{N}), \mu_C)\) measure space. Let \(E_n = \left\lbrace n, n+1, n+2, \ldots\right\rbrace\). In this case \(\mu_C (E_n) = +\infty, E_{n+1} \subseteq E_n \forall \; n\), but \(\bigcap_n E_n = \emptyset \Longrightarrow \mu\left(\bigcap_n E_n\right) = 0 \)
\begin{theorem}[\(\sigma\)-subadditivity of the measure]
\((X, \mathcal{M}, \mu)\) is a measure space. \(\forall \left\lbrace E_n \right\rbrace \subseteq \mathcal{M}\) (not necessarily disjoint): \(\mu\left(\bigcup_n E_n\right) \leq \sum_n \mu(E_n)\)
\end{theorem}
\begin{proof}
    \(E_1, E_2 \in \mathcal{M}\) and also \(E_1 \cup E_2 = E_1 \cup (E_2 \backslash E_1)\) disjoint sets.
    \[
        \mu(E_1 \cup E_2) = \mu(\underbrace{E_2 \backslash E_1}_{\subseteq E_2}) \overset{(monotonicity)}{\leq} \mu(E_1) + \mu(E_2)
    \]
    that means that we have the subadditivity for finitely many sets.
    \[
        A = \bigcup_{n=1}^{\infty} E_n, \quad A_k = \bigcup_{n = 1}^{k} E_n
    \]
    \[
        \left\lbrace A_k \right\rbrace \nearrow, \; A_{k+1} \supseteq A_k, \; \lim_k A_k = A
    \]
    \[
        \mu\left(\bigcup_{n = 1}^{\infty} E_n\right) \overset{(continuity)}{=} \lim_k \mu(A_k) = \lim_k \mu \left(\bigcup_{n=1}^{k} E_n\right) \leq
        \]
        \[
            \leq \lim_k \sum_{n=1}^k \mu(E_n) = \sum_{n = 1}^{\infty} \mu(E_n)
        \]
\end{proof}
Exercise: \((X, \mathcal{M})\) measurable space. \(\mu : \mathcal{M} \to [0, +\infty]\) s.t. \(\mu\) is finitely additive, \(\sigma\)-subadditive and \(\mu(\emptyset) = 0\) \(\Longrightarrow\) \(\mu\) is \(\sigma\)-additive, and hence is a measure.  

Exercise: the Borel-Cantelli lemma states that, given \((X, \mathcal{M}, \mu)\) and \(\left\lbrace E_n \right\rbrace \subseteq \mathcal{M}\). Then
\[
    \sum_{n=0}^{\infty} \mu(E_n) < \infty \Longrightarrow \mu(\limsup_n E_n) = 0
\]
It can be phrased as: \begin{quote}
    If the series of the probability of the events \(E_n\) is convergent, then the probability that \(\infty\)-many events occur is \(0\)
\end{quote}
\begin{proof}
    The thesis is: \[\mu(\limsup_n E_n) = \mu\biggl(\bigcap_{n=1}^{\infty} \underbrace{\bigcup_{k \geq n} E_k}_{A_n := \bigcup_{k\geq n}E_k}\biggr)\]
    Is it true that \(\left\lbrace A_n \right\rbrace \searrow\)? Yes.
    \[
         A_{n+1} = \bigcup_{k \geq n+1}  E_k \subseteq \bigcup_{k \geq n} E_k = A_n
    \]
    Does some \(A_n\) have a finite measure? 
    \[
        \mu(A_n) = \mu\left(\bigcup_{k \geq n} E_k\right) \leq \sum_{k \geq n} \mu(E_k) < \infty
    \]
    by assumption. Therefore, we can use the continuity along decreasing sequences: 
    \[
        \mu(\limsup_n E_n) = \lim_n \mu(A_n) = \lim_n \mu \left(\bigcup_{k \geq n} E_k\right) \overset{\sigma-sub.}{\leq} \lim_n \sum_{k=n}^{\infty} \mu(E_k) = 0
    \]
\end{proof}
\subsubsection*{Sets of \(0\) measure}
\((X, \mathcal{M}, \mu)\) measure space.
\begin{itemize}
    \item \(N \subseteq X\) is a set of \(0\) measure if \(N \in \mathcal{M}\) and \(\mu(N) = 0\)
    \item \(E \subseteq X\) is called \textbf{negligible set} if \(\exists \; N \in \mathcal{M}\) with \(0\) measure s.t. \(E \subseteq N\) (\(E\) does not necessarily stays in \(\mathcal{M}\))
\end{itemize} 
\begin{definition}
    \((X, \mathcal{M}, \mu)\) measure space s.t. every negligible set is measurable (and hence of \(0\) measure), then \(\measurespace\) is said to be a \textbf{complete measure space}.

    A measure space may not be complete. However, let 
    \[
        \overline{\mathcal{M}} := \left\lbrace E \subseteq X : \exists\; F, G \in \mathcal{M} \mbox{ with } F\subseteq E \subseteq G \mbox{ and } \mu(G\backslash F) = 0\right\rbrace
    \]
    Clearly \(\mathcal{M} \subseteq \overline{\mathcal{M}}\). For \(E \in \overline{\mathcal{M}}\), take \(F\) and \(G\) as above and let \(\bar{\mu}(E) = \bar{\mu}(F)\) then \(\bar{\mu}\vert_{\mathcal{M}} = \mu\), and moreover:
\end{definition}
\begin{theorem}
    \(\measurespace\) is a complete measure space. Let's observe that \(\bar{\mu}\) is well defined: let \(E \subseteq X\) and \(F_1,F_2, G_1, G_2\) s.t. \(F_i \subset E \subset G_i \quad i = 1,2\). Then \(\mu(G_i\backslash F_i) = 0\). Now we have to check that \(\mu(F_1) = \mu(F_2)\). 

    Since \[
        F_1 \backslash F_2 \subseteq E\backslash F_2 \subseteq G_2 \backslash F_2
    \] 
    and \(G_2 \backslash F_2\) has \(0\) measure \(\Longrightarrow \mu(F_1 \backslash F_2) = 0\). Then \(F_1 = (F_1 \backslash F_2) \cup (F_1 \cap F_2) \Longrightarrow \mu(F_1) = \mu(F_1 \cap F_2).\) In the same way, \(\mu(F_2) = \mu(F_1 \cap F_2)\)
\end{theorem}
