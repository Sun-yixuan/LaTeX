\section{Lesson 19/10/2022}
\subsection*{The first fundamental theorem of calculus}

Consider \(f \in L^1\left([a,b]\right)\). We can define the \textbf{integral function}
\[F(x) = \int_{[a,b]} f d\lambda = \int_a^b f(t)dt , \quad x \in [a,b]\]

If \(f \in \mathcal{C}\left(\left[a, b\right]\right)\), then \(F\) is differentiable on \(\left[a, b\right]\), and \(F'(x)=f(x)\)

What happens if \(f \in L^1([a, b])\)?



\begin{definition}
    Given \(f \in L^1([a,b])\). We say that \(x \in [a,b]\) is a \textbf{Lebesgue point} for \(f\) if 
    \[
        \lim_{h \to 0} \frac{1}{h} \int_x^{x+h} \abs{f(t) - f(x)} \, dt = 0
    \]
    If \(x=a\) or \(x=b\), this is the left/right \(\lim\).
\end{definition}
\begin{remark}
    A point \(x\) is called a Lebesgue point for \(f\) if \(f\) `does not oscillate too much' close to \(x\):
    \begin{itemize}
        \item \(f\) \(\mathcal{C}([a,b]) \to \text{ every } x \in [a,b]\) is a Lebesgue point.
        \item \[
            f(x) = \begin{cases}
                1  & x > 0 \\
            0 & x < 0
            \end{cases}
        \]
        \[
            \lim_{h \to 0} \frac{1}{h} \int_0^{h} \abs{f(t) - f(0)} \, dt = \lim_{h \to 0} \frac{1}{\abs{h}} \int_0^{h} \abs{0 - 1} \, dt = 0
        \]
    \end{itemize}
\end{remark}
\begin{theorem}[Lebesgue]
    If \(f \in L^1([a.b])\) then a.e. \(x \in [a,b]\) is a Lebesgue point for \(f\)
\end{theorem}
\begin{remark}
    In the definition of Lebesgue point, the pointwise values of \(f\) are relevant 
    \[
        f = g \in L^1 \Longleftrightarrow f = g \text{a.e.}
    \]
    Then the Lebesgue point of \(f\) could be different from the one of \(g\).  
    This is not a big problem if \(f = g\) a.e. on \([a,b] \Longrightarrow f = g \in [a,b]\setminus N\) where \(\lambda(N) = 0\); \(x\) is a Lebesgue point for \(f\), \(\forall \; x \in \left[a, b\right] \setminus M, \ \lambda(M)=0\) \\
    \(\Rightarrow x \) is a Lebesgue point for \(g\), \(\ \forall x \in \left[a, b\right] \setminus (M \cup N)\) \\
    \(\left[a, b\right] \setminus (M \cup N)\) is a set of full measure of Lebesgue points for \(f\) and \(g\).
\end{remark}
To speak about Lebesgue points, one has to choose a specific representative \(f \in L^1([a,b])\). If you change representative, you obtain the same set of Lebesgue points up to sets with \(0\)-measure.

\begin{theorem}[First fundamental theorem of calculus]
Given \(f \in L^1([a,b]),\ F(x) = \int_a^xf(t) \, dt\) \\
Then \(f\) is differentiable a.e. on \([a,b]\) and \(F'(x) = f(x) \text{ a.e. in } [a,b]\)    
\end{theorem}
\begin{proof}
    Let \(x \in [a,b]\) for any Lebesgue point for \(f\) (a.e. \(x \in [a,b]\) is fine). Consider
    \[
        \left|{\frac{F(x+h)-F(x)}{h} - f(x)}\right| = \abs{\frac{1}{h} \int_x^{x+h} (f(t) - f(x)) \,dt } \leq \frac{1}{h} \int_x^{x+h} |f(t) - f(x)| \,dt \to 0 
    \]
    Since \(x\) is a Lebesgue point.
\end{proof}
\begin{definition}
    Given \(f : I \to \real\) is called \textbf{absolutely continuous} in \(I\), \(f \in AC(I)\), if \(\forall \; \epsilon >0 \; \exists \; \delta >0\)  
    s.t. 
    \[
        \bigcup_{k=1}^n [a_x, b_x] \in I \text{ disjoint ma per finta}
    \] 
    \[
        \lambda(\bigcup_{k=1}^n [a_x, b_x]) = \sum_{k=1}^n (b_x -a_x) < \delta
    \]
    \[
        \Rightarrow \sum_{k=1}^n \abs{f(b_k)-f(a_k)} < \epsilon
    \]
\end{definition}
\begin{remark}
    \(f\) is uniformly continuous on \([a,b]\) if \(\forall \; \epsilon > 0\) \(\exists \; \delta > 0\) s.t. 
    \[
        \abs{t - \tau} < \delta \Rightarrow \abs{f(t)-f(\tau)} < \epsilon
    \]
    An absolutely continuous function is also uniformly continuous. \\
    But the converse is false.
\end{remark}

\begin{itemize}
    \item If \(f\) is Lipschitz on \([a,b] \Longrightarrow f \in \text{AC}([a,b])\) 
\end{itemize}
Recall that \(f \in \mbox{Lip}([a,b])\) if \(\exists \; L > 0\) s.t. 
\[
    \abs{f(x) - f(y)} \leq L\abs{x-y} \qquad \forall x, y \in \left[a, b\right]
\]
\underline{Check}: For any \(\epsilon > 0\), and consider 
\[
    \sum_{k=1}^n \abs{f(b_k)-f(a_k)} \leq \sum_{k=1}^n L(b_k-a_k) = L \sum_{k=1}^n (b_k-a_k)
\]
If we take \(\delta = \delta(\epsilon) = \frac{\epsilon}{L}\), then 
\[
    \sum_{k=1}^n (b_x, a_x) < \delta \Longrightarrow \sum_{k=1}^n \abs{f(b_k)-f(a_k) } \leq L \sum_{k=1}^n (b_k-a_k)
\]
\qed
\[
    \mbox{Lip}([a,b]) \subsetneqq \mbox{AC}([a,b]) \subsetneqq \mbox{UC}([a,b])
\]
\begin{theorem}[Regularity of integral functions]
    Given \(f \in L^1([a,b]), F(x) = \int_a^xf(t) \, dt\), then \(F \in \mbox{AC}([a,b])\)
\end{theorem}
To prove the theorem we need the
\begin{theorem}[Absolute continuity of the integral]
    Given \(f \in L^1(X, \mathcal{M}, \mu)\). Then \(\forall \; \epsilon >0 \) \(\exists \; \delta > 0\) s.t.     
    \[
        \begin{array}{l}
            E \in \mathcal{M} \\
            \mu(E) < \delta
        \end{array} 
        \Longrightarrow \int_E \abs{f} \, d\mu < \epsilon
    \]
\end{theorem}
\begin{proof}
    We fix \(\epsilon > 0\). Let \(F_n := \left\{ \abs{f} < n \right\}\), \(n \in \mathbb{N}\). Also \(F_n \in \mathcal{M} \forall \; n\), \(F_n \subseteq F_{n+1}\) and 
    \[
        \bigcup_{n=1}^{\infty} F_n = \left\{ \abs{f} < \infty \right\} =: F
    \]
    \(f \in L^1 \Longrightarrow \abs{f}\) is finite a.e.: \(\mu(X \setminus F) =0\).
    Therefore:
    \[
        \int_X \abs{f} \, d\mu = \int_{X \setminus F} \abs{f} \, d\mu + \int_F \abs{f} \, d\mu 
        = \lim_{n \to\infty} \int_{F_n} \abs{f} \, d\mu
    \]
    \[
        \lim_{n\to\infty} \int_X \abs{f} \left(\- \chi_{F_n} \right) \, d\mu =0
    \]
    \(\forall \; \epsilon >0 \exists \; \bar{n} \in \mathbb{N}\) s.t. 
    \[
        n > \bar{n} \Rightarrow \abs{\int_X \abs{f} \chi_{F_n^C} \, d\mu} < \frac{\epsilon}{2}
    \]
    Now, fix \(\epsilon >0 \), and take \(n > \bar{n}(\epsilon)\). If \(E \in \mathcal{M}\), then
    \[
        \int_E \abs{f} \, d\mu = \int_{E \cap F_n} \abs{f} \, d\mu + \int_{E \cap F_n^C} \abs{f} \, d\mu
        \leq n \int_E 1 \, d\mu + \int_{F_n^C} \abs{f} \, d\mu 
    \]
    If we suppose that \(\mu(E) < \frac{\epsilon}{2n} =: \delta(\epsilon)\), we deduce that
    \[
        n \int_E 1 \, d\mu = n \mu(E) < \frac{\epsilon}{2}
    \]
    Also, since \(n > \bar{n}\)
    \[
        \int_{F_n^C} \abs{f} \, d\mu < \frac{\epsilon}{2}
    \]
    \[
        \Rightarrow \int_E \abs{f} \, d\mu < \epsilon \]
\end{proof}
\begin{proof}[Regularity of integral functions]
    Let \(\epsilon > 0\), and \(\delta = \delta(\epsilon) > 0\) be the value given by the absolute continuity of \(\int \abs{f} \, d\mu\). 
    Take 
    \[
        E = \bigcup_{k=1}^n \left[ a_k, b_k \right] \qquad E \subseteq \left[a, b\right]
    \]
    If \(\lambda(E) < \delta\), then
    \[
        \sum_{k=1}^n \abs{F(b_k) - F(a_k)} = \sum_{k=1}^n \abs{\int_{a_k}^{b_k} f(t)\, dt} 
        \leq \sum_{k=1}^n \int_{a_k}^{b_k} \abs{f(t)}\, dt
        = \int_E \abs{f} \, d\lambda < \epsilon
    \]
    by absolute continuity of \(\int\) 
\end{proof}
\begin{remark}
    \(\sqrt(x)\) is AC\((\left[0, 1\right])\), but is not Lip\((\left[0, 1\right])\).
    \[
        \sqrt{x} = \int_0^x \frac{1}{2\sqrt{t}} \, dt
    \]
    \(\Rightarrow \sqrt{x}\) is the \(\int\) function of a \(L^1\) function \\
    \(\Rightarrow \sqrt{x} \in \text{AC}(\left[0, 1\right])\)
\end{remark}

To sum up: the \(\int \) function of a \((L^1\) function is AC, it is differentiable a.e., and 
\[
    F(x) - F(a) = \int_a^x F'(t) \, dt \tag*{FC}
\]

Suppose \(G\) is differentiable a.e. on \(\left[a, b\right]\) and FC holds for \(G\):
\[
    G(x) - G(a) = \int_a^x G'(t) \, dt
\]
What can we say about \(G\)?
\begin{remark}
    If \(G\) \(\in \mathcal{C}^1(\left[a, b\right]) \Rightarrow\) FC holds. \\
    If FC holds, then \(G' \in L^1(\left[a, b\right]) \) (necessary condition) \\
    Is the necessary condition also sufficient? In general not. 
    Take \(v(x)\), the Vital Cantor function: \(v \in \mathcal{C}([0,1]), v(0)=0, v(1)=1\). \(v\) is differentiable a.e. on \([0,1]\) but the calculus formula doesn't hold!
\end{remark}
\begin{remark}
    A function which is differentiable a.e. on an interval can behave very badly
\end{remark}
\begin{theorem}
    \(G \in \text{AC}([a, b])\). Then \(G\) is differentiable a.e. on \([a, b]\), \(G' \in L^1([a, b])\), and FC holds.
\end{theorem}
\begin{remark}
    These theorems say that AC function are precisely the ones for which FC holds:
    \begin{itemize}
        \item \(G \in \) AC \(\Rightarrow\) FC holds. 
        \item If FC holds, then \(G' \in L^1 ([a, b])\) 
    \end{itemize}
    \(\Rightarrow \int_a^x G'(t) \, dt \in \) AC \\
    \(\Rightarrow G(x) - G(a) = \int_a^x G'(t) \, dt \in \) AC   
\end{remark}
\begin{remark}
    \(v \in \) UC \(([0, 1])\) by continuity and Heine Cantor, but \(v \notin \) AC \(([0, 1])\) because FC does not hold.
\end{remark}

The proof of the second fundamental theorem of calculus is divided into two steps. 
\begin{lemma}
    The second fundamental theorem hold under the additional assumption that \(G\) is monotone.
\end{lemma}
Second step: to get rid of the monotonicity. \\
For step 2, is it useful to give the
\begin{definition}
    \(\left[a, b\right] \subset \mathbb{R}\). Let 
    \[
        \mathcal{P}_{\left[a, b\right]} := \lbrace (x_0, x_1, \dots, x_n): n \in \mathbb{N} \text{ and } a=x_0 < x_1 < x_2 < \dots < x_n = b \rbrace
    \]
    For \(P \in \mathcal{P}_{\left[a, b\right]}\) and \(f: [a, b] \to \barreal\), define
    \[
        v_a^b(f, P) := \sum_{k=1}^n \abs{f(x_k) - f(x_{k-1})}
    \]
    The total variation of \(f \) on \([a, b]\) is 
    \[
        V_a^b (f) := \sup_{P \in \mathcal{P}_{\left[a, b\right]}} v_a^b (f, P)
        = \sup \lbrace \sum_{k=1}^n \abs{f(x_k)-f(x_{k-1})}:n \in \mathbb{N}, a=x_0 < x_1 < \dots <x_n = b \rbrace
    \]
\end{definition}
If \(V_a^b(f) < \infty\), we say that \(f\) is a function with bounded variation, \(f \in \) BV \(([a, b])\)