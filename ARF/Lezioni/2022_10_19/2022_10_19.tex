\section{Lesson 19/10/2022}
\subsection*{The first fundamental theorem of calculus}

Consider \(f \in L^1\left([a,b]\right)\). We can define the \textbf{integral function}
\[F(x) = \int_{[a,b]} f d\lambda = \int_a^b f(t)dt , \quad x \in [a,b]\]

If the function cose 

What happens if ?

\begin{definition}
    Given \(f \in L^1([a,b])\). We say that \(x \in [a,b]\) is a \textbf{Lebesgue point} for \(f\) if 
    \[
        \lim_{h \to 0} \frac{1}{h} \int_x^{x+h} \abs{f(t) - f(x)} \, dt = 0
    \]
    If \(x=a\) or \(x=b\), this is the left/right \(\lim\).
\end{definition}
\begin{remark}
    A point \(x\) is called a Lebesgue point for \(f\) if \(f\) `does not oscillate too much' close to \(x\):
    \begin{itemize}
        \item \(f\) \(\mathcal{C}([a,b]) \to \text{ every } x \in [a,b]\) is a Lebesgue point.
        \item \[
            f(x) = \begin{cases}
                1  & x > 0 \\
            0 & x < 0
            \end{cases}
        \]
        \[
            \lim_{h \to 0} \frac{1}{h} \int_0^{h} \abs{f(t) - f(0)} \, dt = \lim_{h \to 0} \frac{1}{\abs{h}} \int_0^{h} \abs{0 - 1} \, dt = 0
        \]
    \end{itemize}
\end{remark}
\begin{theorem}[Lebesgue theorem]
    If \(f \in L^1([a.b])\) then a.e. \(x \in [a,b]\) is a Lebesgue point for \(f\)
\end{theorem}
\begin{remark}
    In the definition of Lebesgue point, the pointwise values of \(f\) are relevant 
    \[
        f = g \in L^1 \Longleftrightarrow f = g \text{a.e.}
    \]
    Then the Lebesgue pint of \(f\) could be different from the one of \(g\).  
    This is not a big problem if \(f = g\) a.e. on \([a,b] \Longrightarrow f = g \in [a,b]\backslash N\) where \(\lambda(N) = 0\). Altre cose da recuperare
\end{remark}
To speak about Lebesgue points, one has to choose a specific representative \(f \in L^1([a,b])\). If you change representative, you obtain the same set of Lebesgue points up to sets with \(0\)-measure.
\begin{theorem}[Fistt fundamental theorem of calculus]
Given \(f \in L^1([a,b]), F(x) = \int_a^xf(t) \, dt\)
Then \(f\) is differentiable a.e. on \([a,b]\) and \(F'(x) = f(x) \text{ a.e. in } [a,b]\)    
\end{theorem}
\begin{proof}
    Let \(x \in [a,b]\) for any Lebesgue point for \(f\) (a.e. \(\x \in [a,b]\) is fine). Consider
    \[
        \abs{\frac{F(x+h)-F(x)}{h} - f(x)} = \abs{\frac{1}{h} \int_x^{x+h} (f(t) - f(x)) \,dt } \leq \frac{1}{h} \int_x^{x+h} (f(t) - f(x)) \,dt \to 0 
    \]
\end{proof}