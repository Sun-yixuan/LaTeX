\section{Lesson 20/10/2022}
Example and comments:
\begin{itemize}
    \item If \(f\) is bounded and monotone \(\Longrightarrow\) \(f \in \mbox{BV}\)
    \[
        V_a^b (f) = \abs{f(b) - f(a)}
    \]
    Note that \(f\) may not be continuous
    cases
    \item \(f \in \mbox{BV}([a,b]) \Longrightarrow f\) is bounded. Indeed
    \[
        \sup_{x \in [a,b]} \abs{f(x)} \leq \abs{f(x)} + V_a^b(f)
    \]
    \item \(f\) is continuous on \([a,b]\) \(\not\Longrightarrow f \in \mbox{BV}([a,b])\)
    \item \(f \in \mbox{BV}([a,b]) \cap \mbox{UC}([a,b]) \not \Longrightarrow f \in \mbox{AC}([a,b])\)
    Given \(v\) a Vitali-Cantor function. \(v\) is bounded and monotone \(\Longrightarrow v \in \mbox{BV}([0,1])\) and \(v \in \mbox{UC}([0,1])\). But \(v \not \in \mbox{AC}([0,1])\)
\end{itemize}
We can now come back to the proof of Lemma 1
\begin{proof}
    \underline{Preliminary result}: \(A \in \real\) open. Then \(A = \bigcup_{n=1}^{\infty}(a_n, b_n)\)
    any open set of \(\real\) is....
    \underline{Preliminary result}: \(f \in \mbox{AC}([a,b]) \Longleftrightarrow \forall \; \epsilon > 0 \exists \; \delta > 0\)
    \dots
    \dots
    We can start with the proof
\end{proof}

Take this \(\delta\). By regularity of \(\lambda\), \(\exists \; A \) open set of \(\left[a, b\right]\)s.t. \(A \supset E\) and \(\lambda(A) < \delta\). 

....

But then 
\[
    \mu (E) \leq \mu(A) = \sum_n \mu (I_n) \text{ since } \mu \text{ is a measure}
\]

We proved that 
\[
    \lambda(E) = 0 \Rightarrow \forall \; \epsilon >0 : \, \mu(E) < \epsilon \Rightarrow \mu(E) =0
\]
So \(\mu << \lambda\). We can apply Radon Nikodym \(\exists \; \oldphi : \left[ a, b\right] \to \left[0, \infty\right]\) s.t. 
\[
    G(x) - G(a) = \int_a^x \oldphi \, d\lambda
\]
Since \(G\) is bounded, then \(\oldphi \in L^1(\left[a, b\right])\)
\[
    G(x)=G(a) + \int_a^x \oldphi(t) \, dt
\]
By the first foundamental theorem of calculus, this is differentiable a.e. 
\[
    G'(x) = \oldphi(x) \text{ a.e. on } \left[a, b\right]
\]
\[
    G'(x) = cose
\]

Now we want to get rid of the additional assumption (monotonicity)

\begin{proof}
    of the second fundamental theorem of calculus in the general case.
    Idea:  \(G \in AC\) 
    we want to write \(G= G_1 + G_2\) where \(G_1 \nearrow \) and \(G_2 \searrow\), both AC. \\
    Then the second fundamental theorem holds for \(G_1 \) and \(G_2\) so it holds for \(G\) by linearity of the integral. 
    \dots
    Clearly, \(G_1+G_2 = G\), \(G_1, G_2\) are AC, by preliminary result n4.
    \dots
    Therefore, 
    \[
        G_1 (y) - G_1(x) = \frac{1}{2} \left(G(y) - G(x) + V_a^y(G) + V_a^x(G)\right) \geq - \abs{G(y) - G(x)} \geq - V_x^y(G)
    \]
    So \(G_1\) is decreasing. In an analogue way, we can prove that \(G_2\) is decreasing.
\end{proof}

\section*{Functional analysis}
Normed spaces and Banach spaces
\begin{definition}
    Given \(X\) vector space, a norm on \(X\) is a function \(\norm{\cdot}\)
\end{definition}

