\section*{Lesson 13/10/2022}
\subsubsection*{Integration on product spaces}
\((X, \mathcal{M}, \mu), (Y, \mathcal{N},\nu)\) measure spaces. \(f : X \times Y \to \bar{\real}\) measurable.

If \(f \geq 0\), then 
\[
    \iint_{X \times Y} f d\mu\otimes d\nu
\]
Goal: obtain a formula of iterated integral like the one in Analysis 2.

\(\forall \bar{x} \in X\) and \(\bar{y} \in Y\)
\[
    cose
\]
\begin{proposition}
    If \(f\) is measurable \(\Rightarrow\) \(f_{\bar{x}}\) is \((\mathcal{N}, \boreal)\)-measurable and \(f_{\bar{y}}\) is \((\mathcal{M}, \mathcal{B}(\bar{\real}))\)-measurable.
    Then we can conclude
    \( \phi : X \to \bar\real \):
    \[
        \phi(x) = \int_Y f_x d\nu = \int_Y f(x,y) d\nu(y)
    \]
    and \(\psi : Y \to \bar{\real}\)
    \[
        \psi(y) = \int_X f_y d\mu = \int_X f(x,y) d\mu(x)
    \]
\end{proposition}
\underline{Questions:} what is the solution of \(\iint_{X \times Y}\)
cose cose
\begin{theorem}[Tonelli's theorem]
    \((X, \mathcal{M}, \mu)\) and \((Y, \mathcal{N}, \nu)\) complete measure spaces and \(\sigma\)-finite.
    Suppose that \(f\) is \((\mathcal{M} \otimes \mathcal{N}, \mathcal{B}(\bar{\real}))\)-measurable and that \(f > 0\) a.e. on \(X \times Y\). Then \(\psi\) and \(\phi\) are measurable and
    \[
        \iint_{X \times Y} f d\mu \otimes d\nu = cose
    \]
    Equally holds also if one of the integrals is \(\infty\).
\end{theorem}
\begin{remark}
    The double integral can be reduced to single integrals, iterated. Moreover we can always change the order of the integrals
    For sign changing functions the situation is more involved.
\end{remark}
\begin{theorem}[Fubini's theorem]
    \((X, \mathcal{M}, \mu)\) and \((Y, \mathcal{N}, \nu)\) complete measure spaces and \(\sigma\)-finite.
    If \(f \in L^1(X \times Y)\), then \(\psi\) and \(\phi\) defined above are measurable, and cose holds, and all the integrals are finite.
\end{theorem}
\underline{Question}: how to check if \(f\in L^1(X \times Y)\)? Typically, to check cosette

If \(\iiiint_{X \times Y} \vert f \vert d\mu \otimes d\nu < \infty\) then we can apply Fubini for \(\iint_{X \times Y} f d\mu \otimes d\nu\)
\begin{remark}
    the proof of Fubini's and Tonelli's theorems is based for the iterated integrals for characteristic functions.
    (Note that \((\mu \otimes \nu)(E) = \int_X ()\) e altre cosette)
\end{remark}
\begin{remark}
    Sometimes double integrals are very useful to compute single integrals.
\end{remark}
Ex: \(\int_{-\infty}^{+\infty}\exp{-x^2} = \sqrt{\pi}\)


\subsection*{The first fundamental theorem of calculus}

Consider \(f \in L^1\left([a,b]\right)\)nWe can define the \textbf{integral function}
\[F(x) = \int_{[a,b]} f d\lambda = \int_a^b f(t)dt , \quad \]

If the function cose 

What happens if ?
