\section{Lesson 05/10/2022}
\begin{theorem}[Monotone Convergence (or Beppo Levi's theorem)]
    \(f_n : X \to [0, +\infty]\) measurable. Suppose that 
    \begin{enumerate}
        \item \(f_n(x) \leq f_{n+1}(x)\) for a.e. \(x \in X\), for every \(n\)
        \item \(f_n \to f\) a.e. on \(X\) 
    \end{enumerate}
    Then 
    \[
        \int_X f \, d\mu = \lim_n \int_X f_n \, d\mu
    \]
\end{theorem}
\begin{corollary}
    \(f_n : X \to [0, +\infty]\) measurable, then 
    \[
        \int_X \left( \sum_{n=0}^{\infty} f_n\right) \, d\mu = \sum_{n=0}^{\infty} \int_X f_n \, d\mu
    \]
\end{corollary}
\begin{theorem}[Approximation with simple functions]
    Given \((X, \mathcal{M})\) measure space, \(f: X \to [0, +\infty]\) measurable, then \(\exists\) a sequence \(\left\{ s_n \right\}\) of simple functions s.t. 
    \[
        0 \leq s_1 \leq \ldots \leq s_n \leq \ldots \leq f \qquad \text{pointwise } \forall \; x \in X
    \]
    and 
    \[
        s_n (x) \to f(x) \qquad \forall \; x \in X \text{as } n \to \infty 
    \]
    Moreover, if \(f\) is bounded, then \(s_n \to f\) uniformly on \(X\) as \(n \to \infty\).
\end{theorem}
\begin{remark}
    There is also
    \[
        \int_X f \, d\mu = \sup \left\{ \int_X s \, d\mu \, \bigg\vert \begin{array}{l}s\text{ is simple} \\ 0 \leq s \leq f \end{array}\right\}
    \]
\end{remark}
But let \(\left\{ s_n \right\}\) be the sequence given by the simple approximation theorem. By monotone convergence 
\[
    \int_X f \, d\mu = \lim_n \int_X s_n \, d\mu
\]
Ex: \(f, g : X \to [0, +\infty]\). Then 
\[
    \int_X (f+g) \, d\mu = \int_X f \, d\mu + \int_X g \, d\mu
\]
\begin{lemma}[Fatou's Lemma]
    Given \(f_n \to [0, +\infty]\) measurable \(\forall \, n\). Then 
    \[
        \int_X (\liminf_n f_n) \, d\mu \leq \liminf_n \int_X f_n \, d\mu
    \]
    In particular, if \(f_n \to f\) a.e. on \(X\).
\end{lemma}
\begin{proof}
    Given that \((\liminf_n f_n)(x) = \lim_n (\underbrace{\inf_{k \geq n} f_k(x)}_{= g_n (x)})\). Now, for every \(x \in X\), \(\left\{ g_n(x) \right\}\nearrow\)
    \[
        g_{n+1}(x) = \inf_{k \geq n+1} f_k(x) \geq \inf_{k \geq n} f_k(x) = g_n (x)
    \]
    Also, \(g_n \geq 0\) on \(X\). Thus, by monotone convergence
    \[
        \int_X \liminf_n f_n \, d\mu = \int_X \lim_n g_n \, d\mu = \lim_n \int_X g_n \, d\mu = \liminf \int_X g_n \, d\mu
    \]
    Now, note that \(g_n (x) = \inf_{k\geq n} f_k(x) \leq f_n(x) \leq \liminf_n \int_X f_n \, d\mu\) 
\end{proof}
\begin{theorem}[\(\sigma\)-additivity of \(\int\)]
    Given \((X, \mathcal{M}, \mu)\) measurable space, \(\Phi : X \to [0, +\infty]\). Define \(\nu(E) = \int_E \Phi \, d\mu\), with \(E \in \mathcal{M}\). \(\nu : \mathcal{M} \to [0, +\infty]\) is a measure. Moreover, let \(f:X \to [0, +\infty]\) measurable
    \[
        \int_X f \, d\nu = \int_X f\Phi \, d\nu \tag*{*}
    \]
\end{theorem}
\begin{proof}
    \underline{\(\nu\) is a measure}:  
    \(\nu(\emptyset) = 0\), since \(\mu\)
\end{proof}
