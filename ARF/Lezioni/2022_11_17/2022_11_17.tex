\section{Lecture 17/11/2022}
Let \(X, Y\) be normed spaces.
\begin{definition}
    \(T: X \to Y\) is called \textbf{open map} if, \(\forall \; A \subset X \mbox{ open }\),the set \(T(A)\subset Y\) is open.
\end{definition}
\begin{remark}
    Recall that \(T\) is continuous on \(X\) if \(T^{-1}(O)\) is open on \(X\), \(\forall \; O \mbox{ open in } Y\).
\end{remark}
\underline{Ex}: \(f(x) : \mbox{constant}\) is continuous, but not open. \(f((a,b)) = \left\{ \mbox{const} \right\}\)
\begin{theorem}[Open map theorem]
    \(X, Y\) Banach spaces. \(T \in \mathcal{L}(X,Y)\) is surjective. Then \(T\) is an open map.
\end{theorem}
\begin{corollary}
    \(X,Y\) Banach spaces, \(T \in \mathcal{L}(X,Y)\) is bijective. Then \(T\) is an isomprphism: \(T^{-1} \in \mathcal{L}(X,Y)\)
\end{corollary}
\begin{proof}
    \begin{itemize}
        \item \(T : Y \to X\) is linear. (Exercise. Hint: Use \(T^{-1} \circ T = \text{Id} + \) linearity of \(T\))
        \item We want now to check that \(T^{-1}\) is continuous on \(Y\): \((T^{-1})^{-1}(O)\) is open in \(Y\), \(\forall \; O\) open in \(X\). We know that \(T\) is an open map thanks to the open map theorem.
        \[
            (T^{-1})^{-1}(O) = \left\{ y \in Y, T^{-1}(y) \in O \right\} = \left\{ y \in Y, T^{-1}(y) = x, \mbox{ for some } x \in O \right\} =
        \]
        \[
            = \left\{ y \in Y, y = Tx, \mbox{ for some } x \in O\right\} = T(O) \mbox{ is open}
        \]
    Since \(T\) is an open map, \(\forall \; O \subset X\), open.
    \end{itemize}
\end{proof}
\begin{corollary}
    \(X\) vector space, \(\normdot, \normdot_*\) norms on \(X\). Assume \((X, \normdot), (X, \normdot_*)\) are Banach spaces. 
    Assume that \(\exists \; C_1 > 0\) s.t. 
    \[
        \norm{x}_* \leq C_1\norm{x} \quad \forall ; x \in X
    \]
    Then \(\normdot\) and \(\normdot_*\) are equivalent, namely \(\exists \; C_2 > 0\) s.t. 
    \[
        \norm{x} \leq C_2 \norm{x}_*
    \]
\end{corollary}
\begin{proof}
    Consider
    \[
        \begin{array}{lrl}
            I :& (X, \normdot) & \to (X, \normdot_*) \\
            &x & \mapsto x
        \end{array}
    \]
    By assumption, \(I\) is bounded: \(\exists \; C_1 > 0\) s.t. 
    \[
        \norm{Ix}_* = \norm{x}_* \leq C_2 \norm{x}
    \]
    \(I\) is bijective.

    Thus, by the corollary before
    \[
        I^{-1} = I \in \mathcal{L}((X, \normdot_*), (X, \normdot))
    \]
    namely \(\exists \; C_2 > 0\) s.t. 
    \[
        \stackbelow{\norm{Ix}}{\norm{x}} \leq C_2 \norm{x}_*
    \]
\end{proof}
\begin{definition}
    \(T : D(T) \subset X \to Y\) linear operator. We say that \(T\) is \textbf{closed} if \(\forall \; \left\{ x_n \right\} \subset D(t)\). 
    \[
        \begin{rcases*}
            x_n \to x & \mbox{in } X \\
            Tx_n \to y & \mbox{in } Y
        \end{rcases*} \Rightarrow x \in D(T) \mbox{ and } Tx = y
    \]

\end{definition}
\underline{Ex}: \(X = Y = \mathcal{C}^0([0,1])\) with the supremum norm.
\[
    T = \frac{d}{dx}
\]
\(T\) is not continuous. But it is closed: it can be proved that if \(\left\{ f_n \right\} \subset \mathcal{C}^1([0,1])\) is s.t.
\[
    \begin{rcases*}
        f_n \to f & \mbox{uniformly} \\
        f_n' \to g & \mbox{uniformly}
    \end{rcases*} \Rightarrow f \mbox{ is } \mathcal{C}^1([0,1]) \mbox{ and } f' = g
\] 
\underline{Ex}: \(T \in \mathcal{L}(X,Y) \Rightarrow T \mbox{ is closed}\)
\begin{remark}
    \(T\) is a closed operator \(\Leftrightarrow\) the graph of \(T\)
    \[
        \mbox{graph}(T) = \left\{ (x, Tx): x \in X \right\} \mbox{ is closed}
    \]
\end{remark}
\begin{theorem}[Closed graph theorem]
    \(X, Y\) Banach spaces. 
    
    \(T : X \to Y\) linear closed operator (\(D(T) = X\)). 
    
    Then \(T \in \mathcal{L}(X,Y)\).
\end{theorem}
\begin{remark}
    In general it is easier to prove that an operator is closed, rather than it it continuous.
\end{remark}
\begin{proof}
    Define on \(X\) the graph-norm of \(T\)
    \[
        \norm{x}_* = \norm{x}_X + \norm{Tx}_Y
    \]
    Then is a norm on \(X\). If \(\left\{ x_n \right\} \in X\) is a Cauchy sequence for \(\normdot_*\), then \(\left\{ x_n \right\}\) is a Cauchy sequence in \((X, \normdot_X)\) and \(\left\{ Tx_n \right\}\) is a Cauchy sequence on \((Y, \normdot_Y)\)
    \[
        \Rightarrow \begin{rcases*}
            x_n \to x & \mbox{in } X \\
            Tx_n \to y & \mbox{in } Y
        \end{rcases*} \mbox{ since } T \mbox{ is closed, we deduce that } y = Tx
    \]
    Thus 
    \[
        \norm{x_n - x}_X + \norm{Tx_n - Tx} \to 0
    \]
    This proves that \((X, \normdot_*)\) is a Banach space. Also, we know that 
    \[
        \norm{x}_X \leq \norm{x}_X + \norm{Tx}_Y = \norm{x}_*
    \]
    By the last corollary of the open map theorem, \(\exists \; C_2\) s.t. 
    \[
        \norm{x}_* \leq C_2 \norm{x_X}
    \]
    \[
        \norm{Tx}_Y \leq \norm{x}_* \leq C_2 \norm{x}_X \quad \forall \; x \in X
    \]
    This means that \(T\) is bounded.
\end{proof}
\subsection*{Dual spaces}
\(X\) normed space: 
\[
    X^* = \mathcal{L}(X, \real) \mbox{ is called \textbf{dual space of} }X
\]
\(X\) normed space, \(Y\) Banach space \(\Rightarrow \mathcal{L}(X, Y)\) is a Banach space with \(\normdot_{\mathcal{L}}\).

Since \(\real\) is a Banach space, the dual space \(X^*\) is a Banach space with 
\[
    \norm{L}_* = \sup_{\norm{x}_X \leq 1} \abs{Lx}
\]
\underline{Ex}: 
\begin{itemize}
    \item In \(\real^n\), only linear functional is separated by a scalar product:
        \[
            L : \real^n \to \real \mbox{ is linear } \Rightarrow \exists ! \; y \in \real^n \mbox{ s.t. } Lx = <y,x>
        \]
        It can be proved that 
        \[
            L \subset (\real^n)^* \mapsto y \in \real^n
        \]
        is an isometric isomorphism 
        \[
            (\real^n)^* \approxeq 
        \]
        Then \(X^*\) is very complicated.
    \item mille cose 
\end{itemize}

\dots

\begin{proposition}
    If \(p \in [1, \infty]\) then \(L_g \in (L^p(X)^*)\). Moreover, 
    \begin{itemize}
        \item if \(p > 1\), then \(\norm{L_g}_* = \norm{g}_{p'}\)
        \item if \(p=1\) then \(\norm{L_g}_* = \norm{g}_{\infty}\) with more assumptions (they are satisfied in \((X, \mathcal{L}(X), \lambda)\))
    \end{itemize}
\end{proposition}

\begin{remark}
    We are saying that \(L^{p'}\) can be identified with a subspace of the dual space \((L^p)^*\) and this identification is an isometry.
\end{remark}

Question: are there functional in \((L^p)^*\)?

\begin{proof}
    (of the proposition)
    \begin{itemize}
        \item Case \(p=\infty\) ex
        \item Case \(p=1\) but difficult it's ok if you don't do it
        \item Case \(p \in (1, \infty)\)
        
        \(L_g\) is clearly linear, by linearity of \(\int\), indeed:
        \(\forall \; \alpha \; \beta \in \real\), \(f_1 \; f_2 \in L^p(X)\). Then
        \[
            ...
        \]
        We want to show now that \(L_g \) is bounded. 
        We proved in (*) that 
        \[
            \abs{L_g f} \leq \norm{g}_{p'} \norm{f}_p \quad \forall \; f \in L^p(\Omega)
        \]
        This shows that \(L_g\) is bounded, with norm  \(\norm{L_g}_* \leq \norm{g}_{p'}\) (remember that \(\norm{T}_\mathcal{L} = \inf \{ M>0: \norm{Tx} _Y \leq M \norm{x}_X \quad \forall x \in X\} \))

        We want to show that \(\norm{L_g}_* = \norm{g}_{p'}\). If \(\norm{L_g}_* < \norm{g}_{p'}\), then \(\exists \; M < \norm{g}_{p'}\) s.t. 
        \[
            \abs{L_g f } \leq M \norm{f}_p \quad \forall \; f \in L^p
        \]
        We rule out this possibility by choosing an explicit \(\tilde{f} \in L^p \) s.t.
        \[
            \abs{L_g \tilde{f} } = \norm{g}_{p'} \norm{\tilde{f}}_*
        \]
        We take 
        \[
            .
        \]
        Now, 
        \[
            \norm{\tilde{f}}_p^p = \int_X \abs{\tilde{f}}^p \, d\mu = \int_X \frac{\abs{g}^{p(p'-1)}}{\norm{g}_p^{p(p'-1)}} \, d\mu = (*)
        \]
        \((p')' = p \Rightarrow p=\frac{p'}{p'-1} \Rightarrow p(p'-1) = p'\)
        \[
            (*) = \frac{1}{\norm{g}_{p'}^{p'}} \int_X \abs{g}^{p'} \, d\mu = \frac{\norm{g}_{p'}^{p'}}{\norm{g}_{p'}^{p'}} = 1
        \]    
        \[
            \abs{L_g \tilde{f}} = \abs{\int_X} \frac{\abs{g}}{\norm{g}_{p'-1}^{p'}} .....
            = \frac{1}{\norm{g}_{p'-1}^{p'}} \norm{g}_{p'}^{p'} = \norm{g}_{p'} = \norm{g}_{p'}\norm{\tilde{f}}_p
        \]
    \end{itemize}
\end{proof}

\subsubsection*{Hahn Banach}

\begin{definition}
    \(X\) vector space. A map \(p: X \to \real\) is called sublinear functional if 
    \begin{itemize}
        \item \(p(\alpha x) = \alpha p(x) \qquad \forall \; x \in X\), \(\alpha >0\)
        \item \(p(x+y) \leq p(x) + p(y) \qquad \forall x, y \in X\)  
    \end{itemize}
\end{definition}

\begin{theorem}[Hahn Banach]
    \(X\) real vector space, \(p: X \to \real\) sublinear functional. 
    \(Y\) subspace of \(X\) and suppose that \(\exists \; f: Y \to \real \) linear on \(Y\) s.t. 
    \[
        f(y) \leq p(y) \quad \forall \, y \in Y
    \]
    Then \(\exists\) a linear functional \(F: X \to \real \) s.t. 
    \[
        F(y) = f(y) \quad \forall \; y \in Y \tag*{\(F\) is an extension of \(f\)}
    \]
    Moreover,
    \[
        F(x) \leq p(x) \quad \forall \; x \in X
    \] 
\end{theorem}
