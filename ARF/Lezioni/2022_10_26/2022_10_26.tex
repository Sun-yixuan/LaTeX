\section{Lesson 26/10/2022}
\((X, \norm{\cdot}) \to (X, d) \to \) open sets, closed sets, bounded sets....

In \(\real^n\) we are used to work with \(\norm{\cdot}_2\), but we could have many different norms.
\begin{definition}
    Let \(\normdot\) and \(\normdot_2\) be two norms on the same vector space \(X\). We say that these norms are \textbf{equivalent} if \(\exists \; m, M >0\) s.t. 
    \[
        m\norm{x} \leq \norm{x} \leq M\norm{x} \quad \forall\; x \in X
    \]
\end{definition}
It can be proved that if two norms are equivalent they lead to different metric spaces, but to the same open sets, closed sets, convergent sequences, compact sets \dots
\begin{theorem}
    If \(X\) is any finite dimension vector space, then all the norms on \(X\) are equivalent.
\end{theorem}
\begin{remark}
    This is why in \(\real^n\) usually one does not specify the choice of the norm. One choose the Euclidean norm, since it comes from a scalar product. (ref. Hilbert spaces)
\end{remark}
\underline{Preliminary fact}: The set \(S_1 = \left\{ s \in \real^n : \norm{x}_1  = 1\right\}\) is compact in \((\real^n, d)\)
\begin{proof}
    We show that any norm is equivalent to \(\normdot_1\)
    \[
        x = \sum_{i=1}^n x_i e_i \qquad \left\{ e_i \right\}_{i= 1,\ldots, n} \mbox{ canonical basis}
    \]
    Let's introduce the norm star 
    \[
        \norm{x}_* = \norm{\sum_{i=1}^n x_i e_i}_* < \sum_{i=1}^n \norm{x_i e_i}_* = \sum_{i=1}^n \abs{x_i} \norm{e_i}_* \leq \left(\max_{1 \leq i \leq n} \norm{e_i}_*\right)
    \]
    We proved that \(\exists \; M> 0\) s.t.
    \[
        \norm{x}_* \leq M \norm{x}_1 \quad \forall \; x \in X
    \]
    Note that this proves that \(\phi(x) = \norm{x}_*\) is continuous in (\(X, d\)) indeed 
    \[
        x_n \to x \Leftrightarrow d_1(x_n, x) \to 0
    \]
    then 
    \[
        \abs{\phi(x_n) - \phi(x)} = \abs{\norm{x_n}_* -\norm{x}} \leq \norm{x_n - x}_* \leq M\norm{x_n - x}_1 \to 0
    \]
    Therefore, by the Weierstrass theorem, \(\exists\) a minimum point \(x_0 \in S_1\) s.t. 
    \[
        \phi(x) \geq \phi(x_n) = m \quad \forall\; x \in S_1
    \]
    (recall that \(S_1\) is compact)
    \[
        \norm{x}_* \geq m \quad \forall \; x \in S_1
    \]
    We claim that \(m>0\). If \(m=0\) then \(\norm{x}_* = 0 \Rightarrow x_0  = 0\) that is impossible.

    Thus \(m>0\). Let now \(y \in \real^n, y \neq 0\). Then 
    \[
        \frac{y}{\norm{y}_1} \in S_1 \Rightarrow \norm{\frac{y}{\norm{y}_1}}_* \geq m \Rightarrow \frac{1}{\norm{y}_1} \norm{y}_* \geq m \Rightarrow \norm{y}_*m\geq m \norm{y}_1 \quad \forall \; y \in \real^n 
    \]
\end{proof}
If \(\dim X = +\infty\), then there are many non-equivalent norms.

\underline{Ex}: In \(\mathcal{C}^0([a,b])\), we can define \(\normdot_{\infty}\) and \(\norm{f}_1 = \int_a^b\abs{f(t)} \, dt\).

This is a norm in \(\mathcal{C}^0\)


\subsubsection*{Separability}
\((X, d)\) metric space. 
\begin{definition}
    We say that \(X\) is separable if \(\exists \; A \in X\) which is dense (\(\bar{A} = X\)) and countable 
\end{definition}
In \(\real^n\), \(\mathbb{Q}^n\) which is dense and countable. In \(\infty-\dim\) we can have separable spaces or not. 

For instance, \((L^{\infty}, \normdot_{\infty})\) is not separable. Instead \((\mathcal{C}^0([a,b]), \normdot_{\infty})\) is a separable space. 

\begin{proof}[Sketch of the proof]We will use the \textbf{Stone-Weierstrass theorem}.

The set of polynomials is dense on \(\mathcal{C}^0([a,b])\) is an uncountable set. However it can be proved that the set of polynomials with coefficients in \(\mathbb{Q}\) is dense in the set of all polynomials

Moreover this set is countable. Then, by Stone-Weierstrass this is a countable dense set in \(\mathcal{C}^0([a,b])\)
\end{proof}
\begin{remark}
    One can show that \(\mathcal{C}^0(K)\) is separable whenever \(K\) is a compact set of a metric space \((X,d)\)
\end{remark}
\subsubsection*{Compactness}
In finite dimension (in \(\real^n\)), one has that
\[
    E \subset X \mbox{ is compact } \Leftrightarrow E \mbox{ is closed and bounded}
\]
If \(\dim X = \infty\), then only `\(\Rightarrow\)' is true. In finite dimension, we know that the closed unit ball is compact
\[
    \bar{B}_1(0) = \left\{ x \in \real^n : \norm{x} \leq 1\right\}
\]
What happens now if \((X, \normdot)\) is on \(\infty-\dim\) normed space?
\begin{theorem}[Riesz's theorem]
    \(X\) normed space, \(\dim X = +\infty\) \(\Rightarrow \bar{B}_1(0)\) is not compact
\end{theorem}
\begin{remark}
    It is well known that if \(E \in \real^n\) is compact, then \(\forall \; \left\{ x_n \right\} \in E \) \(\exists \; \left\{ x_{n_k} \right\}\) subsequence s.t. \(x_{n_k} \to x \in E\).  
    This proposition is much harder to prove in \(\infty-\dim\).
\end{remark}
The proof of the Riesz's theorem is based on the Riesz's \textbf{quasi-orthogonality lemma}.
\begin{lemma}
    Let \(X\) be a normed space, \(E \subsetneq X\) a closed subspace. Then \(\forall \; \epsilon \in [0,1]\)
\end{lemma}