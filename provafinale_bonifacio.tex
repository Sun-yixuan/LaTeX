\documentclass[a4paper,12pt]{article}
\usepackage{amssymb}
\usepackage{amsmath}
\usepackage{hhline}
\usepackage{ragged2e}
\usepackage{physics}
\usepackage{bm}
\usepackage[margin=2cm]{geometry}
\usepackage{bigfoot}
\usepackage{amsthm}
\usepackage{tikz}
\usepackage{tabularx}
\usepackage{graphicx}
\usetikzlibrary{shapes.geometric, arrows}
\tikzstyle{startstop} = [rectangle, rounded corners, minimum width=3cm, minimum height=1cm,text centered, draw=black, fill=red!30]
\tikzstyle{io} = [trapezium, trapezium left angle=70, trapezium right angle=110, minimum width=3cm, minimum height=1cm, text centered, draw=black, fill=blue!30]
\tikzstyle{process} = [rectangle, minimum width=3cm, minimum height=1cm, text centered, draw=black, fill=orange!30]
\tikzstyle{decision} = [diamond,aspect = 2, text centered, draw=black, fill=green!30]
\tikzstyle{arrow} = [thick,->,>=stealth]
\usepackage{newunicodechar}
\newunicodechar{≠}{\ensuremath{\not =}}
\usepackage{textcomp}
\usepackage[makeroom]{cancel}
\pagenumbering{gobble}
\newlength\mylength
\setlength\mylength{0.1cm}
\newcolumntype{Y}{>{\Centering\arraybackslash}X}

\newtheoremstyle{break}
  {\partopsep}{\topsep}%  
  {\normalfont}{}
  {\bfseries}{}%
  {\newline}{}%
  \theoremstyle{break}
\newtheorem{theorem}{Teorema}[section]
\newtheorem{corollary}{Corollario}[section]
\newtheorem{proposition}{Proposizione}[section]
\renewcommand*{\proofname}{\textbf{Dimostrazione}}
\renewcommand\qedsymbol{$\bigstar$}
\newtheorem{definition}{Definizione}[section]

\let\oldemptyset\emptyset
\let\emptyset\varnothing

\newcommand{\ind}{\perp\!\!\!\!\perp} 
\newcommand{\code}[1]{\texttt{#1}}
\newcommand{\xdownarrow}[1]{%
  {\left\downarrow\vbox to #1{}\right.\kern-\nulldelimiterspace}
}
\newcommand{\xuparrow}[1]{%
  {\left\uparrow\vbox to #1{}\right.\kern-\nulldelimiterspace}
}
\newcommand{\arrvline}{\hfil\kern\arraycolsep\vline\kern-\arraycolsep\hfilneg}

\long\def\symbolfootnotemark[#1]#2{\begingroup%
\def\thefootnote{\fnsymbol{footnote}}\footnotetext[#1]{#2}\footnotemark[#1]\endgroup}

\long\def\symbolfootnotetext[#1]#2{\begingroup%
\def\thefootnote{\fnsymbol{footnote}}\footnotetext[#1]{#2}\endgroup}


\numberwithin{equation}{section}





\begin{document}
\title{Elaborato finale}
\author{Andrea Bonifacio - 910626}
\date{}
\maketitle

L'arrivo della pandemia causata dal virus SARS-CoV-2 ha instillato una serie di dubbi riguardo l'efficacia dei metodi di prevenzione attualmente in uso. Inoltre l'aumento di malattie infettive negli anni recenti, come il virus Ebola o Zika, ha reso necessario un miglioramento dei modelli matematici utilizzati per descrivere un'epidemia, affinché gli enti governativi e chi si occupa di salute pubblica avesse degli strumenti adeguati per rispondere a queste problematiche.  

I modelli matematici in grado di rappresentare l'evoluzione di una malattia infettiva si sono sempre rivelati strumenti estremamente efficaci per mettere in atto politiche volte a proteggere la popolazione dall'infezione. Solitamente vengono classificati in base a come gestiscono la variabilità, il tempo o come strutturano la popolazione. La maggior parte di questi modelli si basa su una divisione del tempo continua, in quanto assumono la presenza di un flusso di dati costante. Tuttavia, nel mondo reale, i dati sono spesso comunicati a intervalli costanti, rendendo necessaria una discretizzazione del modello a tempo continuo.  

In questo caso si prende in esempio il modello Kermack-McKendrick, un sistema di equazioni differenziali presentato nel 1927, e si trasforma in un modello a tempo discreto, più compatibile con le necessità di chi si occupa di modellistica matematica, in quanto, per esempio, è molto più semplice da utilizzare con l'ausilio di strumenti informatici. Per descrivere questo modello si suddivide la popolazione presa in oggetto in tre categorie: Suscettibili, Infetti
e Recuperati. 

La variazione di persone all'interno delle tre categorie del modello era stata inizialmente formulata come un sistema di equazioni differenziali a tempo continuo.  
Inoltre sono state necessarie una serie di assunzioni prima di procedere alla parametrizzazione e creazione del modello: la popolazione in oggetto non varia di numero, ossia non ci sono nascite, morti, immigrazioni ed emigrazioni al suo interno. Inoltre chi guarisce dalla malattia ha una immunità totale e non può essere infettato nuovamente.  
Nel modello viene impostato un parametro che indica quante persone un malato può infettare, nel caso tale parametro abbia valore minore di uno, il grafico che rappresenta l'andamento del numero di persone infette sarà una curva che decresce in maniera monotona, mentre in caso sia maggiore di uno, il grafico avrà una forma "a campana", che rappresenta molto bene l'andamento di molte malattie infettive. Un altro vantaggio della discretizzazione del modello Kermack-McKendrick, è il fatto che un processo a tempo discreto può essere concettualizzato come una evoluzione attraverso un insieme di eventi.  

Tuttavia, si può facilmente intuire come un modello del genere possa avere forti limitazioni, in quanto le assunzioni fatte prima della sua costruzione sono fortemente incompatibili con quello che accade nel mondo reale. Inoltre il modello non ha modo di tenere conto delle contromisure che possono essere messe in atto, come lo sviluppo di vaccini o l'utilizzo di politiche di distanziamento sociale. Si può sicuramente trovare un modo di implementare delle modifiche al modello per renderlo in grado di reagire meglio ai cambiamenti che potrebbero essere messi in atto e renderlo uno strumento molto efficace nello studio delle pandemie.

\end{document}